\documentclass[reqno]{article}
\usepackage{amsmath}
\usepackage{amssymb}
\usepackage{amsthm}
\usepackage{mathrsfs}
\usepackage{enumerate}
\usepackage{esint}
\usepackage{relsize}
\usepackage{graphicx}
\usepackage{float}
\usepackage{amsthm}
\usepackage{tikz}
\usepackage{stmaryrd}
\usetikzlibrary{graphs}

\setlength{\textheight}{9truein}
\setlength{\topmargin}{-0.5truein}
\setlength{\textwidth}{6truein}
\setlength{\oddsidemargin}{.25truein}
\setlength{\parskip}{6pt plus 2pt minus 1pt}

\newcommand{\Pic}[1]{\text{Pic}(#1)}
\newcommand{\Div}[1]{\text{Div}(#1)}
\newcommand{\divv}[1]{\text{div}(#1)}
\newcommand{\Z}{\mathbb{Z}}
\newcommand{\Jac}{\text{Jac}}

\newtheorem{lemma}{Lemma}
\newtheorem*{theorem}{Theorem}
\newtheorem*{conjecture}{Theorem}

\setlength{\parindent}{40pt}

\begin{document}
	\title{Nematohydrodynamics, Quasi-2D (Stream Function) and Linear Approximations}
	\author{Lucas Myers}
	\maketitle
	
	\section*{Assumptions}
	We will consider a coupled nematic liquid crystal and hydrodynamic system on a flat substrate. We assume the fluid velocity and Q-tensor magnitude are small so we neglect terms higher than order-1 in $Q_{ij}$ and $v_i$. Additionally, we assume no acceleration so that $\partial v_i/\partial t = 0$ always. We give an initial director field configuration of $\mathbf{n} = (\cos\phi, \sin\phi, 0)$ for some $\varphi$ (to be specified) so that the Q-tensor, given by $Q_{ij} = S/2(3n_i n_j - \delta_{ij})$ takes the form:
	\begin{align}
		Q_{ij} &= \frac{S}{2}\left[
			\begin{matrix}
				3\cos^2\varphi - 1 & 3\cos\varphi\sin\varphi & 0 \\
				3\cos\varphi\sin\varphi & 3\sin^2\varphi - 1 & 0 \\
				0 & 0 & -1
			\end{matrix}
		\right]\\
		&= \frac{S}{2}\left[
			\begin{matrix}
				3\cos^2\varphi - 1 & \frac{3}{2}\sin2\varphi & 0 \\
				\frac{3}{2}\sin2\varphi & 3\sin^2\varphi - 1 & 0 \\
				0 & 0 & -1
			\end{matrix}
		\right]
	\end{align}
	
	\section{Three-dimensional linearized equations}
	
	\subsection{Computing homoegenous and elastic generalized force}
	From Svensek and Zumer, the free energy density is given by
	\begin{equation}
		f = \phi(Q) + \frac{1}{2} L \partial_i Q_{jk} \partial_i Q_{jk}
	\end{equation}
	so that the homogeneous elastic part of the generalized force is given by:
	\begin{equation}
		h^{he}_{ij} = L\partial_k^2 Q_{ij} - \frac{\partial \phi}{\partial Q_{ij}} + \lambda \delta_{ij} + \lambda_k\epsilon_{kij}
	\end{equation}
	where $\phi$ is the Landau de Gennes bulk free energy:
	\begin{equation}
		\phi(Q) = \frac{1}{2}A Q_{ij}Q_{ji} + \frac{1}{3}B Q_{ij}Q_{jk}Q_{ki} + \frac{1}{4} C(Q_{ij}Q_{ji})^2
	\end{equation}
	We begin by explicitly calculating the second term in terms of $Q_{ij}$, one term at a time in $\phi$:
	\begin{equation}
	\begin{split}
		\frac{\partial}{\partial Q_{mn}}\left(\frac{1}{2}AQ_{ij}Q_{ji}\right) &= \frac{1}{2}A \left(\delta_{im}\delta_{jn}Q_{ji} + Q_{ij}\delta_{jm}\delta_{in}\right) \\
		&= \frac{1}{2}A (Q_{nm} + Q_{nm}) \\
		&= A Q_{mn}
	\end{split}
	\end{equation}
	where in the last step we've used symmetry of $Q_{ij}$.
	\begin{equation}
	\begin{split}
		\frac{\partial}{\partial Q_{mn}} \left( \frac{1}{3}BQ_{ij}Q_{jk}Q_{ki} \right) &= \frac{1}{3}B ( \delta_{im}\delta_{jn}Q_{jk}Q_{ki} + Q_{ij}\delta_{jm}\delta_{kn}Q_{ki} + Q_{ij}Q_{jk}\delta_{mk}\delta_{ni} ) \\
		&= \frac{1}{3}B(Q_{nk}Q_{km} + Q_{im}Q_{ni} + Q_{nj}Q_{jm}) \\
		&= BQ_{ni}Q_{im}
	\end{split}
	\end{equation}
	And the final term gives:
	\begin{equation}
	\begin{split}
		\frac{\partial}{\partial Q_{mn}}\left( \frac{1}{4} C(Q_{ij}Q_{ji})^2 \right) &= \frac{1}{4}C \cdot 2(Q_{ij}Q_{ji}) \frac{\partial(Q_{kl}Q_{lk})}{\partial Q_{mn}} \\
		&= \frac{1}{4}C \cdot 2(Q_{ij}Q_{ji}) \cdot (\delta_{mk}\delta_{nl}Q_{lk} + Q_{kl}\delta_{lm}\delta_{kn}) \\
		&= \frac{1}{4}C \cdot 2(Q_{ij}Q_{ji}) \cdot (Q_{nm} + Q_{nm}) \\
		&= C Q_{mn} (Q_{ij}Q_{ji})
	\end{split}
	\end{equation}
	Thus, the total homogeneous and elastic force reads:
	\begin{equation}
		h^{he}_{ij} = L\partial^2 Q_{ij} - A Q_{mn} - BQ_{ni}Q_{im} - C Q_{mn} (Q_{ij}Q_{ji}) + \lambda\delta_{ij} + \lambda_k \epsilon_{kij}
	\end{equation}
	
	\subsection{Computing viscous force explicitly}
	Alrighty then, now we need the viscous force on the liquid crystals. From Svensek and Zumer, the viscous force is given by:
	\begin{equation}
		-h^v_{ij} = \frac{1}{2} \mu_2 A_{ij} + \mu_1 N_{ij}
	\end{equation}
	with 
	\begin{equation}
		N_{ij} = \frac{d Q_{ij}}{dt} + W_{ik} Q_{kj} - Q_{ik} W_{kj}
	\end{equation}
	and
	\begin{equation}
		\frac{d Q_{ij}}{dt} = \frac{\partial Q_{ij}}{\partial t} + (v\cdot \nabla)Q_{ij}
	\end{equation}
	The second two terms in the expression for $N_{ij}$ are quadratic in $v_i$ and $Q_{ij}$ so we may drop them, and $(v\cdot \nabla)Q_{ij}$ is clearly quadratic. Hence, we make the approximation
	\begin{equation}
		N_{ij} \approx \frac{\partial Q_{ij}}{\partial t}
	\end{equation}
	We also have the definition
	\begin{equation}
		A_{ij} = (\partial_i v_j + \partial_j v_i)
	\end{equation}
	Note that we want to restrict our analysis to an incompressible fluid, so we must have a further restriction that:
	\begin{equation}\label{eq:incompressible}
		\partial_i v_i = 0
	\end{equation}
	One way of implicitly enforcing this requirement in two dimensions is to define a stream function $\psi_i$ such that $\epsilon_{ijk}\partial_j \psi_k = v_i$ -- note that for $v_i$ defined on a simply-connected set satisfying \eqref{eq:incompressible} this is always true. For a two-dimensional flow, we have $v_3 = 0$ always, so that 
	\begin{equation}
	\begin{split}
		v_3 &= \partial_1 \psi_2 - \partial_2 \psi_1 = 0 \\
		v_2 &= \partial_3 \psi_1 - \partial_1 \psi_3 \\
		v_1 &= \partial_2 \psi_3 - \partial_3 \psi_1
	\end{split}
	\end{equation}
	Suppose we set $\psi_2 = \psi_1 = 0$. Then the condition on $v_3$ is satisfied, and the expression becomes
	\begin{equation}
	\begin{split}
		v_2 &= -\partial_1 \psi_3 \\
		v_1 &= \partial_2 \psi_3
	\end{split}
	\end{equation}
	which implies that
	\begin{equation}
		\psi_3 = \int v_2 dx + f(y)
	\end{equation}
	giving
	\begin{equation}
		v_1 = \int \frac{\partial v_2}{\partial y} dx + \frac{df}{dy}
	\end{equation}
	We may solve this equation by integrating in $y$. Thus, for any $v_1$ and $v_2$ we may produce a $\psi_i = (0, 0, \psi_3)$ which describes the 2D flow. Then the expression for the complete viscous force is
	\begin{equation}
	\begin{split}
		-h^v_{ij} &= \frac{1}{2}\mu_2 (\partial_i v_j + \partial_j v_i) + \mu_1 \frac{\partial Q_{ij}}{\partial t} \\
		&= \frac{1}{2}\mu_2 (\epsilon_{jkl} \partial_i \partial_k \psi_l +  \epsilon_{ikl} \partial_j \partial_k \psi_l) + \mu_1 \frac{\partial Q_{ij}}{\partial t}
	\end{split}
	\end{equation}
	Note that this equation is simpler than one might think, since both of the terms in parentheses only yield nonzero results when $l = 3$. 
	One equation of motion is then given by 
	\begin{equation}
		h^{he}_{ij} + h^{v}_{ij} = 0
	\end{equation}
	or explicitly
	\begin{multline}
		\mu_1 \frac{\partial Q_{ij}}{\partial t} = L\partial^2 Q_{ij} - A Q_{mn} - BQ_{ni}Q_{im} - C Q_{mn} (Q_{ij}Q_{ji}) + \lambda\delta_{ij} + \lambda_k \epsilon_{kij} - \frac{1}{2}\mu_2 (\epsilon_{jkl} \partial_i \partial_k \psi_l +  \epsilon_{ikl} \partial_j \partial_k \psi_l)
	\end{multline}
	Using this, we may update $Q_{ij}$ in time by solving for $\partial Q_{ij}/\partial t$ in terms of $Q_{ij}$ and $v_i$ from the previous iteration.
	
	\subsection{Computing the elastic stress tensor explicitly}
	The elastic stress tensor is obtained via
	\begin{equation}
		\sigma^{e}_{ij} = -\frac{\partial f}{\partial (\partial_i Q_{kl})} \partial_j Q_{kl}
	\end{equation}
	Note that only the elastic part of the free energy make references to derivatives:
	\begin{equation}
	\begin{split}
		\frac{\partial f}{\partial (\partial_i Q_{kl})} &= \frac{\partial}{\partial (\partial_i Q_{kl})} \frac{1}{2} L \partial_j Q_{mn} \partial_j Q_{mn} \\
		&= \frac{1}{2} L (\delta_{ij}\delta_{km}\delta_{ln} \partial_j Q_{mn} + \partial_j Q_{mn} \delta_{ij}\delta_{km}\delta_{ln}) \\
		&= \frac{1}{2} L (\partial_i Q_{kl} + \partial_i Q_{kl}) \\
		&= L\partial_i Q_{kl}
	\end{split}
	\end{equation}
	Then the elastic stress tensor is given by
	\begin{equation}
		\sigma^e_{ij} = -L\partial_i Q_{kl} \partial_j Q_{kl}
	\end{equation}
	
	\subsection{Computing viscous stress tensor explicitly}
	The viscous stress tensor is given by
	\begin{equation}
		\sigma^v_{ij} = \beta_1 Q_{ij}Q_{kl}A_{kl} + \beta_4 A_{ij} + \beta_5 Q_{ik} A_{ki} + \frac{1}{2} \mu_2 N_{ij} - \mu_1 Q_{ik}N_{kj} + \mu_1 Q_{jk}N_{ki}
	\end{equation}
	However, only the $\beta_4$ and $\mu_2$ are linear in $Q_{ij}$ and $v_i$. Hence, this simplifies to
	\begin{equation}
		\sigma^v_{ij} \approx \beta_4 A_{ij} + \frac{1}{2} \mu_2 N_{ij}
	\end{equation}
	Again, plugging in for $A_{ij}$ and $N_{ij}$ as we did for the viscous force, we get
	\begin{equation}
		\sigma^v_{ij} \approx \beta_4 (\partial_i v_j + \partial_j v_i) + \frac{1}{2} \mu_2  \frac{\partial Q_{ij}}{\partial t}
	\end{equation}
	
	\subsection{Computing the fluid equation of motion}
	The equation of motion for the fluid is given by
	\begin{equation}
		\rho \frac{\partial v_i}{\partial t} = -\partial_i p + \partial_j (\sigma^v_{ji} + \sigma^{e}_{ji})
	\end{equation}
	We've made the assumption that $\partial v_i/\partial t \approx 0$. Plugging in for $\sigma^v_{ji}$ and $\sigma^e_{ji}$ yields
	\begin{equation}
	\begin{split}
		0 &= -\partial_i p + \partial_j \left( -L\partial_j Q_{kl} \partial_i Q_{kl} + \beta_4 (\partial_j v_i + \partial_i v_j) + \frac{1}{2} \mu_2  \frac{\partial Q_{ji}}{\partial t} \right) \\
		&= -\partial_i p - L\left[(\partial^2 Q_{kl}) \partial_i Q_{kl} + (\partial_j Q_{kl}) \partial_j \partial_i Q_{kl}\right] + \beta_4(\partial^2 v_i + \partial_j \partial_i v_j) + \frac{1}{2} \mu_2 \partial_j \frac{\partial Q_{ji}}{\partial t} \\
		&= -\partial_i p - L\left[(\partial^2 Q_{kl}) \partial_i Q_{kl} + (\partial_j Q_{kl}) \partial_j \partial_i Q_{kl}\right] + \beta_4\partial^2 v_i + \frac{1}{2} \mu_2 \partial_j \frac{\partial Q_{ji}}{\partial t}
	\end{split}
	\end{equation}
	where in the last step we have used the condition that $\partial_j v_j = 0$. Let's take the curl in order to get rid of the pressure term:
	\begin{equation}
	\begin{split}
		0 &= -\epsilon_{mni}\partial_n\partial_i p - L\epsilon_{mni}\partial_n\left[(\partial^2 Q_{kl}) \partial_i Q_{kl} + (\partial_j Q_{kl}) \partial_j \partial_i Q_{kl}\right] + \beta_4 \epsilon_{mni}\partial_n \partial^2 v_i + \frac{1}{2}\mu_2\epsilon_{mni}\partial_n \partial_j\frac{\partial Q_{ji}}{\partial t} \\
	\end{split}
	\end{equation}
	Now note that the first term becomes
	\begin{equation}
	\begin{split}
		\epsilon_{mni}\partial_n\partial_i p &= -\epsilon_{min}\partial_n\partial_i p \\
		&= -\epsilon_{min}\partial_i\partial_n p \\
		&= -\epsilon_{mni}\partial_n\partial_i p
	\end{split}
	\end{equation}
	where in the last step we've reindexed $n \leftrightarrow i$. Hence, the first term vanishes. For the second term we find:
	\begin{multline}
		L\epsilon_{mni}\partial_n\left[(\partial^2 Q_{kl}) \partial_i Q_{kl} + (\partial_j Q_{kl}) \partial_j \partial_i Q_{kl}\right] = L\epsilon_{mni} \left[ (\partial_n \partial^2 Q_{kl})\partial_i Q_{kl} + (\partial^2 Q_kl)\partial_n \partial_i Q_{kl} \right. \\ \left. + (\partial_n\partial_j Q_{kl})\partial_j\partial_i Q_{kl} + (\partial_j Q_{kl})\partial_j\partial_i\partial_n Q_{kl} \right]
	\end{multline}
	Note that the second, third, and fourth terms are symmetric in $n$ and $i$ so that they go to zero with the Levi-Civita (as with the pressure term). Our equation of motion becomes
	\begin{equation}
		L\epsilon_{mni}(\partial_n \partial^2 Q_{kl})\partial_i Q_{kl} = \beta_4 \epsilon_{mni}\partial_n \partial^2 v_i + \frac{1}{2} \mu_2 \epsilon_{mni} \partial_n \partial_j \frac{\partial Q_{ji}}{\partial t}
	\end{equation}
	Lastly, we may plug in the stream function expression for velocity. This yields
	\begin{equation}
	\begin{split}
		\beta_4 \epsilon_{mni}\partial_n\partial^2 v_i &= \beta_4 \epsilon_{mni}\epsilon_{ijk}\partial^2\partial_n\partial_j \psi_k \\
		&= \beta_4 (\delta_{mj}\delta_{nk} - \delta_{mk}\delta_{nj}) \partial^2\partial_n\partial_j \psi_k \\
		&= \beta_4 \partial^2 \left( \partial_n \partial_m \psi_n - \partial_n \partial_n \psi_m \right) \\
		&= \beta_4 \left( \partial_m \partial_n \partial^2 \psi_n - \partial^4 \psi_m \right)
	\end{split}
	\end{equation}
	Hence, the final equation of motion becomes:
	\begin{equation}
		L\epsilon_{mni}(\partial_n \partial^2 Q_{kl})\partial_i Q_{kl} = \beta_4 \left( \partial_m \partial_n \partial^2 \psi_n - \partial^4 \psi_m \right) + \frac{1}{2} \mu_2 \epsilon_{mni} \partial_n \partial_j \frac{\partial Q_{ji}}{\partial t}
	\end{equation}

	\section{Two-dimensional linearized equations}
	In two dimensions we define $\psi$ by $v_i = \epsilon_{ij} \partial_j \psi$. Additionally, $Q_{ij}$ is defined by $Q_{ij} = S(2n_i n_j - \delta_{ij})$ so that it is two-dimensional and traceless, and our symmetry condition is given by a $\lambda \epsilon_{ij}$ term. Thus, our force equation becomes:
	\begin{equation} \label{eq:2DForce}
		\mu_1 \frac{\partial Q_{ij}}{\partial t} = L\partial^2 Q_{ij} - A Q_{mn} - B Q_{ni}Q_{im} - C Q_{mn}(Q_{ij}Q_{ji}) + \lambda_1 \delta_{ij} + \lambda_2 \epsilon_{ij} - \frac{1}{2}\mu_2(\epsilon_{jk}\partial_i\partial_k \psi + \epsilon_{ik}\partial_j \partial_k \psi)
	\end{equation}
	Now to take the curl of the fluid equation of motion, we operate with $\epsilon_{im}\partial_m$ on either side:
	\begin{equation}
		0 = -\epsilon_{im}\partial_m\partial_i p -  L\epsilon_{im}\partial_m \left[(\partial^2 Q_{kl}) \partial_i Q_{kl} + (\partial_j Q_{kl}) \partial_j \partial_i Q_{kl}\right] + \beta_4 \epsilon_{im}\partial_m\partial^2 v_i + \frac{1}{2} \mu_2 \epsilon_{im}\partial_m \partial_j \frac{\partial Q_{ji}}{\partial t}
	\end{equation}
	The first term, as before, gives
	\begin{equation}
	\begin{split}
		\epsilon_{im}\partial_m\partial_i p &= -\epsilon_{mi}\partial_m \partial_i p \\
		&= -\epsilon_{mi}\partial_i\partial_m p \\
		&= -\epsilon_{im}\partial_m\partial_i p
	\end{split}
	\end{equation}
	Hence, the first term vanishes. The first term in brackets gives
	\begin{equation}
	\begin{split}
		\epsilon_{im}\partial_m\left(\partial^2 Q_{kl}\right)\partial_i Q_{kl} &= \epsilon_{im}\left( \left( \partial_m \partial^2 Q_{kl} \right)\partial_i Q_{kl} + \left( \partial^2 Q_{kl} \right)\partial_m\partial_i Q_{kl} \right) \\
		&= \epsilon_{im} \left(\partial_m \partial^2 Q_{kl}\right)\partial_i Q_{kl}
	\end{split}
	\end{equation}
	where the second term vanishes because $\epsilon_{im}$ is antisymmetric in $im$ and $\partial_m\partial_i$ is symmetric in $im$. The second term in brackets reads
	\begin{equation}
	\begin{split}
		\epsilon_{im}\partial_m\left( \left( \partial_j Q_{kl} \right) \partial_j \partial_i Q_{kl} \right) &= \epsilon_{im} \left( \left( \partial_j\partial_m Q_{kl} \right)\partial_j\partial_i Q_{kl} + \left( \partial_j Q_{kl} \right) \partial_j\partial_i\partial_m Q_{kl} \right) \\
		&= 0
	\end{split}
	\end{equation}
	where both terms vanish for the same reason as above. For the third term we invoke the stream function to give
	\begin{equation}
	\begin{split}
		\epsilon_{im}\partial_m\partial^2 v_i &= \epsilon_{im}\epsilon_{in} \partial_m\partial_n \partial^2 \psi \\
		&= \delta_{mn} \partial_m \partial_n \partial^2 \psi \\
		&= \partial^4 \psi
	\end{split}
	\end{equation}
	where we have used that
	\begin{equation}
	\begin{split}
		\epsilon_{im}\epsilon_{in} &= -\epsilon_{mi}\epsilon_{in} \\
		&= -\begin{pmatrix} 0 & 1 \\ -1 & 0 \end{pmatrix} \begin{pmatrix} 0 & 1 \\ -1 & 0 \end{pmatrix} \\
		&= -\begin{pmatrix} -1 & 0 \\ 0 & -1 \end{pmatrix} \\
		&= \delta_{mn}
	\end{split}
	\end{equation}
	Note that, by $\partial^4$ we mean $\partial_m\partial_m \partial_n\partial_n$. Hence, our equation of motion reads
	\begin{equation} \label{eq:biharmonic}
		L\epsilon_{im}\left( \partial_m \partial^2 Q_{kl} \right)\partial_i Q_{kl} = \beta_4 \partial^4 \psi + \frac{1}{2} \mu_2 \epsilon_{im} \partial_m \partial_j \frac{\partial Q_{ji}}{\partial t}
	\end{equation}
	It looks like it might actually be totally reasonable to get these equations in terms of $S$ and $\varphi$. 
	
	\section{Formulation in terms of $S$ and $\varphi$}
	
	\subsection{Partial derivatives of $Q_{kl}$}
	We seek to explicitly write out $\partial_i \partial_j \partial_k Q_{mn}$ explicitly in terms of derivatives of $Q_{mn}$ with respect to $S$ and $\varphi$, and spatial derivatives of $S$ and $\varphi$. Then we won't have to deal with the Lagrange multiplier scheme. 
	\begin{equation}\label{eq:chainrule}
	\begin{split}
		\partial_i\partial_j\partial_k Q_{mn} &= \partial_i\partial_j\left(\frac{\partial S}{\partial x_k} \frac{\partial Q_{mn}}{\partial S} + \frac{\partial \varphi}{\partial x_k} \frac{\partial Q_{mn}}{\partial \varphi}\right) \\
		&= \partial_i\left( \frac{\partial^2 S}{\partial x_j \partial x_k} \frac{\partial Q_{mn}}{\partial S} + \frac{\partial S}{\partial x_k} \frac{\partial}{\partial x_j} \frac{\partial Q_{mn}}{\partial S} + \frac{\partial^2 \varphi}{\partial x_j \partial x_k} \frac{\partial Q_{mn}}{\partial \varphi} + \frac{\partial \varphi}{\partial x_k} \frac{\partial}{\partial x_j}\frac{\partial Q_{mn}}{\partial \varphi} \right) \\
		&= \partial_i\left( \frac{\partial^2 S}{\partial x_j \partial x_k} \frac{\partial Q_{mn}}{\partial S} + \frac{\partial S}{\partial x_k} \left( \frac{\partial S}{\partial x_j} \frac{\partial^2 Q_{mn}}{\partial S^2} + \frac{\partial \varphi}{\partial x_j} \frac{\partial^2 Q_{mn}}{\partial \varphi \partial S} \right) + \frac{\partial^2 \varphi}{\partial x_j \partial x_k} \frac{\partial Q_{mn}}{\partial \varphi}\right. \\ 
		&\qquad\qquad+ \left.\frac{\partial \varphi}{\partial x_k} \left( \frac{\partial S}{\partial x_j} \frac{\partial^2 Q_{mn}}{\partial \varphi \partial S} + \frac{\partial \varphi}{\partial x_j} \frac{\partial^2 Q_{mn}}{\partial \varphi^2} \right) \right) \\
		&= \partial_i \left( \frac{\partial^2 S}{\partial x_j \partial x_k}\frac{\partial Q_{mn}}{\partial S} + \frac{\partial^2 \varphi}{\partial x_j\partial x_k} \frac{\partial Q_{mn}}{\partial \varphi} + \frac{\partial S}{\partial x_k}\frac{\partial S}{\partial x_j} \frac{\partial^2 Q_{mn}}{\partial S^2} + \frac{\partial S}{\partial x_k} \frac{\partial \varphi}{\partial x_j} \frac{\partial^2 Q_{mn}}{\partial \varphi \partial S} \right. \\
		&\qquad\qquad \left.+ \frac{\partial \varphi}{\partial x_k} \frac{\partial S}{\partial x_j} \frac{\partial^2 Q_{mn}}{\partial \varphi \partial S} + \frac{\partial \varphi}{\partial x_k}\frac{\partial \varphi}{\partial x_j} \frac{\partial^2 Q_{mn}}{\partial \varphi^2}\right)
	\end{split}
	\end{equation}
	At this point, I think the best scheme is to run the calculation term by term -- we can simplify at the end if possible. First term:
	\begin{equation} \label{eq:expand-begin}
	\begin{split}
		\partial_i \left( \frac{\partial^2 S}{\partial x_j \partial x_k} \frac{\partial Q_{mn}}{\partial S} \right) &= \frac{\partial^3 S}{\partial x_i \partial x_j \partial x_k} \frac{\partial Q_{mn}}{\partial S} + \frac{\partial^2 S}{\partial x_j \partial x_k} \left( \frac{\partial S}{\partial x_i} \frac{\partial^2 Q_{mn}}{\partial S^2} + \frac{\partial \varphi}{\partial x_i} \frac{\partial^2 Q_{mn}}{\partial \varphi \partial S} \right)
	\end{split}
	\end{equation}
	Second term:
	\begin{equation}
	\begin{split}
		\partial_i \left( \frac{\partial^2 \varphi}{\partial x_j \partial x_k} \frac{\partial Q_{mn}}{\partial \varphi} \right) &= \frac{\partial^3 \varphi}{\partial x_i \partial x_j \partial x_k}\frac{\partial Q_{mn}}{\partial \varphi} + \frac{\partial^2 \varphi}{\partial x_j \partial x_k} \left( \frac{\partial S}{\partial x_i} \frac{\partial ^2 Q_{mn}}{\partial S \partial \varphi} + \frac{\partial \varphi}{\partial x_i}\frac{\partial^2 Q_{mn}}{\partial \varphi^2} \right)
	\end{split}
	\end{equation}
	Third term:
	\begin{equation}
	\begin{split}
		\partial_i \left( \frac{\partial S}{\partial x_k} \frac{\partial S}{\partial x_j}\frac{\partial^2 Q_{mn}}{\partial S^2} \right) &= \frac{\partial^2 S}{\partial x_i \partial x_k} \frac{\partial S}{\partial x_j}\frac{\partial^2 Q_{mn}}{\partial S^2} + \frac{\partial S}{\partial x_k} \frac{\partial^2 S}{\partial x_k \partial x_j} \frac{\partial^2 Q_{mn}}{\partial S^2}\\
		&\qquad\qquad + \frac{\partial S}{\partial x_k}\frac{\partial S}{\partial x_j} \left( \frac{\partial S}{\partial x_i} \frac{\partial^3 Q_{mn}}{\partial S^3} + \frac{\partial \varphi}{\partial x_i} \frac{\partial^3 Q_{mn}}{\partial \varphi \partial S^2} \right)
	\end{split}
	\end{equation}
	Fourth term:
	\begin{equation}
	\begin{split}
		\partial_i\left( \frac{\partial S}{\partial x_k}\frac{\partial \varphi}{\partial x_j} \frac{\partial^2 Q_{mn}}{\partial \varphi \partial S} \right) &= \frac{\partial^2 S}{\partial x_i \partial x_k} \frac{\partial \varphi}{\partial x_j} \frac{\partial^2 Q_{mn}}{\partial \varphi \partial S} + \frac{\partial S}{\partial x_k} \frac{\partial^2 \varphi}{\partial x_i \partial x_j} \frac{\partial^2 Q_{mn}}{\partial \varphi \partial S} \\
		&\qquad\qquad + \frac{\partial S}{\partial x_k}\frac{\partial \varphi}{\partial x_j} \left( \frac{\partial S}{\partial x_i}\frac{\partial^3 Q_{mn}}{\partial S \partial \varphi \partial S} + \frac{\partial \varphi}{\partial x_i}\frac{\partial^3 Q_{mn}}{\partial \varphi^2 \partial S} \right)
	\end{split}
	\end{equation}
	Fifth term (note that it is the same as the fourth term, except $S \leftrightarrow \varphi$):
	\begin{equation}
	\begin{split}
		\partial_i\left( \frac{\partial \varphi}{\partial x_k}\frac{\partial S}{\partial x_j} \frac{\partial^2 Q_{mn}}{\partial \varphi \partial S} \right) &= \frac{\partial^2 \varphi}{\partial x_i \partial x_k} \frac{\partial S}{\partial x_j} \frac{\partial^2 Q_{mn}}{\partial \varphi \partial S} + \frac{\partial \varphi}{\partial x_k} \frac{\partial^2 S}{\partial x_i \partial x_j} \frac{\partial^2 Q_{mn}}{\partial \varphi \partial S} \\
		&\qquad\qquad + \frac{\partial \varphi}{\partial x_k}\frac{\partial S}{\partial x_j} \left( \frac{\partial \varphi}{\partial x_i}\frac{\partial^3 Q_{mn}}{\partial \varphi \partial S \partial \varphi} + \frac{\partial S}{\partial x_i}\frac{\partial^3 Q_{mn}}{\partial S^2 \partial \varphi} \right)
	\end{split}
	\end{equation}
	Sixth term (note that it is the same as the third term except $\varphi \leftrightarrow S$):
	\begin{equation} \label{eq:expand-end}
	\begin{split}
		\partial_i \left( \frac{\partial \varphi}{\partial x_k} \frac{\partial \varphi}{\partial x_j}\frac{\partial^2 Q_{mn}}{\partial \varphi^2} \right) &= \frac{\partial^2 \varphi}{\partial x_i \partial x_k} \frac{\partial \varphi}{\partial x_j}\frac{\partial^2 Q_{mn}}{\partial \varphi^2} + \frac{\partial \varphi}{\partial x_k} \frac{\partial^2 \varphi}{\partial x_k \partial x_j} \frac{\partial^2 Q_{mn}}{\partial \varphi^2}\\
		&\qquad\qquad + \frac{\partial \varphi}{\partial x_k}\frac{\partial \varphi}{\partial x_j} \left( \frac{\partial \varphi}{\partial x_i} \frac{\partial^3 Q_{mn}}{\partial \varphi^3} + \frac{\partial S}{\partial x_i} \frac{\partial^3 Q_{mn}}{\partial S \partial \varphi^2} \right)
	\end{split}
	\end{equation}
	I think I will be able to cut down on the number of terms when we plug in, because some will go to zero immediately, and some will be able to be grouped together when we have the repeated derivatives. 
	
	\subsection{Explicit expression of $Q_{ij}$ in terms of the director field}
	It seems like there are two natural ways to define $Q_{ij}$ in two dimensions. One  way is to just truncate the 3rd row and column of the regular $3D$ Q-tensor and then rescale so that it is still traceless -- i.e. $Q_{ij} = S(2n_i n_j - \delta_{ij})$. Alternatively, there is a suggestion from Cortese, Eggers, and Liverpool to define it as
	\begin{equation}
		Q_{ij} = S
		\begin{pmatrix}
		2n_x^2 - 1 & 2n_x n_y \\
		2n_x n_y & 1 - 2n_x^2
		\end{pmatrix}
	\end{equation}
	It is unclear to me which one ends up being more useful in which situations. For now I think I will stick to the first definition. 
	
	\subsection{Solving for matrix elements of the derivatives of $Q_{ij}$ explicitly}
	Note that we may always exchange partial derivative order, and that $\partial/\partial S$ always has the effect of stripping off a factor of $S$. Furthermore, $\partial^2/\partial S^2$ makes everything zero. Hence, we only care about the $\varphi$ derivatives. First, no derivatives:
	\begin{equation}
	\begin{split}
		Q_{ij} &= S
		\begin{pmatrix}
			2\cos^2\varphi - 1 & \sin2\varphi \\
			\sin2\varphi & 2\sin^2\varphi - 1
		\end{pmatrix} \\
		&= S
		\begin{pmatrix}
			\cos2\varphi & \sin2\varphi \\
			\sin2\varphi & -\cos2\varphi
		\end{pmatrix}
	\end{split}
	\end{equation}
	Then first derivative:
	\begin{equation}
	\begin{split}
		\frac{\partial Q_{ij}}{\partial \varphi} &= S
		\begin{pmatrix}
			-4\cos\varphi\sin\varphi & 2\cos2\varphi \\
			2\cos2\varphi & 4\sin\varphi\cos\varphi
		\end{pmatrix} \\
		&= S
		\begin{pmatrix}
			-2\sin2\varphi & 2\cos2\varphi \\
			2\cos2\varphi & 2\sin2\varphi
		\end{pmatrix}\\
		&= 2S
		\begin{pmatrix}
		-\sin2\varphi & \cos2\varphi \\
		\cos2\varphi & \sin2\varphi
		\end{pmatrix}
	\end{split}
	\end{equation}
	Then the second derivative:
	\begin{equation}
		\frac{\partial^2 Q_{ij}}{\partial \varphi^2} = 4S
		\begin{pmatrix}
			-\cos2\varphi & -\sin2\varphi \\
			-\sin2\varphi & \cos2\varphi
		\end{pmatrix}
	\end{equation}
	And the third derivative:
	\begin{equation}
		\frac{\partial^3 Q_{ij}}{\partial \varphi^3} = 8S
		\begin{pmatrix}
			\sin2\varphi & -\cos2\varphi \\
			-\cos2\varphi & -\sin2\varphi
		\end{pmatrix}
	\end{equation}
	
	\subsection{Solving for $Q_{ij}$ contractions}
	$B$ term in the force equation:
	\begin{equation}
	\begin{split}
		Q_{ni}Q_{im} &= S
		\begin{pmatrix}
			\cos2\varphi & \sin2\varphi \\
			\sin2\varphi & -\cos2\varphi
		\end{pmatrix}
		S
		\begin{pmatrix}
			\cos2\varphi & \sin2\varphi \\
			\sin2\varphi & -\cos2\varphi
		\end{pmatrix} \\
		&= S^2
		\begin{pmatrix}
			\cos^22\varphi + \sin^22\varphi & \cos2\varphi\sin2\varphi - \sin2\varphi\cos2\varphi \\
			\cos2\varphi\sin2\varphi - \sin2\varphi\cos2\varphi & \sin^22\varphi + \cos^22\varphi
		\end{pmatrix}\\
		&= S^2\delta_{mn}
	\end{split}
	\end{equation}
	$C$ term in the force equation:
	\begin{equation}
		Q_{ij}Q_{ji} = S^2\left( \cos^2 2\varphi + \sin^2 2\varphi + \sin^2 2\varphi + \cos^2 2\varphi \right) = 2S^2
	\end{equation}
	so that
	\begin{equation}
		Q_{mn}Q_{ij}Q_{ji} = 2S^3
		\begin{pmatrix}
			\cos2\varphi & \sin2\varphi \\
			\sin2\varphi & -\cos2\varphi
		\end{pmatrix}
	\end{equation}
	
	\subsection{Explicitly solving for the time evolution of $S$ and $\varphi$ independently}
	From \eqref{eq:2DForce} we have $\partial Q_{ij}/\partial t$ expressed as a sum of tensor contractions and derivatives. We would like to explicitly solve for $\partial S/\partial t$ and $\partial \varphi/\partial t$. To do this, note that
	\begin{equation}
	\begin{split}
		\frac{\partial Q_{ij}}{\partial t} &= \frac{\partial S}{\partial t} \frac{\partial Q_{ij}}{\partial S} + \frac{\partial \varphi}{\partial t} \frac{\partial Q_{ij}}{\partial \varphi} \\
		&= \frac{\partial S}{\partial t}
		\begin{pmatrix}
			2\cos^2\varphi - 1 & \sin2\varphi \\
			\sin2\varphi & 2\sin^2\varphi - 1
		\end{pmatrix}
		+ 2S\frac{\partial \varphi}{\partial t}
		\begin{pmatrix}
			-\sin2\varphi & \cos2\varphi \\
			\cos2\varphi & \sin2\varphi
		\end{pmatrix}
	\end{split}
	\end{equation}
	Note that only the first two components are actually free. This gives us two equations and two unknowns, so we should be able to solve for the time evolution. These equations read
	\begin{align}
		\frac{\partial Q_{11}}{\partial t} &= \frac{\partial S}{\partial t}\cos2\varphi - \frac{\partial \varphi}{\partial t} 2S\sin2\varphi
		\label{eq:timeeq1} \\
		\frac{\partial Q_{12}}{\partial t} &= \frac{\partial S}{\partial t} \sin2\varphi + \frac{\partial \varphi}{\partial t} 2S\cos2\varphi \label{eq:timeeq2}
	\end{align}
	Solving \eqref{eq:timeeq1} for $\partial S/\partial t$ and plugging it into \eqref{eq:timeeq2} yields
	\begin{equation}
	\begin{split}
		\frac{\partial Q_{12}}{\partial t} &= \tan2\varphi\left( \frac{\partial Q_{11}}{\partial t} + \frac{\partial \varphi}{\partial t} 2S\sin2\varphi \right) + \frac{\partial \varphi}{\partial t} 2S \cos2\varphi \\
		&= \tan2\varphi \frac{\partial Q_{11}}{\partial t} + \frac{\partial \varphi}{\partial t}2S\left( \tan2\varphi\sin2\varphi + \cos2\varphi \right) \\
		\implies \cos2\varphi \frac{\partial Q_{12}}{\partial t} &= \sin2\varphi \frac{\partial Q_{11}}{\partial t} + \frac{\partial \varphi}{\partial t} 2S\left( \sin^22\varphi + \cos^22\varphi \right) \\
		\implies \frac{\partial \varphi}{\partial t} &= \frac{1}{2S}\left( \cos2\varphi \frac{\partial Q_{12}}{\partial t} - \sin2\varphi \frac{\partial Q_{11}}{\partial t} \right)
	\end{split}
	\end{equation}
	Now we have an expression for the time evolution of $\varphi$ which is valid for all but the completely isotropic state (which makes sense, I suppose). We may plug this back into \eqref{eq:timeeq1} to find the time evolution of $S$
	\begin{equation}
	\begin{split}
		\frac{\partial Q_{11}}{\partial t} &= \frac{\partial S}{\partial t} \cos2\varphi - \left( \cos2\varphi \frac{\partial Q_{12}}{\partial t} - \sin2\varphi \frac{\partial Q_{11}}{\partial t} \right)\sin2\varphi \\
		&= \frac{\partial S}{\partial t}\cos2\varphi - \cos2\varphi\sin2\varphi \frac{\partial Q_{12}}{\partial t} + \sin^22\varphi \frac{\partial Q_{11}}{\partial t} \\
		&= \frac{\partial S}{\partial t}\cos2\varphi - \cos2\varphi\sin2\varphi \frac{\partial Q_{12}}{\partial t} +  \frac{\partial Q_{11}}{\partial t} - \cos^22\varphi \frac{\partial Q_{11}}{\partial t}\\
		\implies \frac{\partial S}{\partial t} &= \sin2\varphi \frac{\partial Q_{12}}{\partial t} + \cos2\varphi \frac{\partial Q_{11}}{\partial t}
	\end{split}
	\end{equation}
	It appears that this expression is always valid. Now we just need to plug in the expression on the right side of \eqref{eq:2DForce} for $\partial Q_{11}/\partial t$ and $\partial Q_{12}/\partial t$. We begin by explicitly writing out the first term of \eqref{eq:2DForce} based on the two-derivative expresssion in \eqref{eq:chainrule} -- here $j = k$
	\begin{equation}
	\begin{split}
		\partial^2 Q_{mn} &= \left( \frac{\partial^2 S}{\partial x_k^2}\frac{\partial Q_{mn}}{\partial S} + \frac{\partial^2 \varphi}{\partial x_k^2}\frac{\partial Q_{mn}}{\partial \varphi} + \left( \frac{\partial S}{\partial x_k} \right)^2 \frac{\partial^2 Q_{mn}}{\partial S^2} + \left( \frac{\partial \varphi}{\partial x_k} \right)^2 \frac{\partial^2 Q_{mn}}{\partial \varphi^2} + 2\frac{\partial S}{\partial x_k}\frac{\partial \varphi}{\partial x_k}\frac{\partial^2 Q_{mn}}{\partial \varphi \partial S} \right) \\
		&= \left(\partial^2 S\right)
		\begin{pmatrix}
			2\cos^2\varphi - 1 & \sin2\varphi \\
			\sin2\varphi & 2\sin^2\varphi - 1
		\end{pmatrix}
		+ \left(\partial^2 \varphi\right)
		2S
		\begin{pmatrix}
			-\sin2\varphi & \cos2\varphi \\
			\cos2\varphi & \sin2\varphi
		\end{pmatrix} \\
		&\qquad\qquad +
		\left(\partial_k \varphi\right)^2
		4S
		\begin{pmatrix}
			-\cos2\varphi & -\sin2\varphi \\
			-\sin2\varphi & \cos2\varphi
		\end{pmatrix}
		+ 4 \left( \partial_k S \:\partial_k \varphi \right)
		\begin{pmatrix}
			-\sin2\varphi & \cos2\varphi \\
			\cos2\varphi & \sin2\varphi
		\end{pmatrix}
	\end{split}
	\end{equation}
	We may then read off the $11$ and $12$ components as
	\begin{align}
		\partial^2 Q_{11} &= \left(\partial^2 S\right) \cos2\varphi -  \left( \partial^2\varphi \right) 2S\sin2\varphi - \left(\partial_k \varphi\right)^2 4S \cos2\varphi - 4\left( \partial_k S\: \partial_k \varphi \right) \sin2\varphi \\
		\partial^2 Q_{12} &= \left( \partial^2 S \right)\sin2\varphi + \left( \partial^2 \varphi \right) 2S\cos2\varphi - \left( \partial_k \varphi \right)^2 4S\sin2\varphi + 4\left( \partial_k S\: \partial_k \varphi \right)\cos2\varphi
	\end{align}
	Reading off the $11$ and $12$ components of the Landau-de Gennes free energy part of the force equation yields:
	\begin{align}
		- A Q_{11} - B Q_{1i}Q_{i1} - C Q_{11}(Q_{ij}Q_{ji}) &= -\left(B + S\cos2\varphi \left(A + 2C S^2 \right)\right) \\
		- A Q_{12} - B Q_{1i}Q_{i2} - C Q_{12}(Q_{ij}Q_{ji}) &= -S\sin2\varphi\left(A + CS^2\right)
	\end{align}
	Finally we can take a look at the stream term. Here we get
	\begin{align}
		\epsilon_{1k}\partial_1 \partial_k \psi + \epsilon_{1k}\partial_1\partial_k \psi &= 2\partial_1\partial_2 \psi \\
		\epsilon_{2k}\partial_1 \partial_k \psi + \epsilon_{1k}\partial_2\partial_k \psi &= 
		-\partial_1\partial_2 \psi + \partial_2\partial_1\psi = 0
	\end{align}
	
	\subsection{Writing down the inhomogeneous biharmonic equation for $\psi$}
	This'll be a doozy. We begin with the left side of \eqref{eq:biharmonic}. We first expand the factor in parentheses using \eqref{eq:expand-begin}-\eqref{eq:expand-end} with $j = k$. This yields
	\begin{equation}
	\begin{split}
		\partial_m\partial^2 Q_{kl} &= \frac{\partial^3 S}{\partial x_m \partial x_j^2} \frac{\partial Q_{kl}}{\partial S} + \frac{\partial^2 S}{\partial x_j^2}\frac{\partial \varphi}{\partial x_m}\frac{\partial^2 Q_{kl}}{\partial \varphi \partial S} + \frac{\partial^3 \varphi}{\partial x_m \partial x_j^2}\frac{\partial Q_{kl}}{\partial \varphi} + \frac{\partial^2 \varphi}{\partial x_j^2} \left( \frac{\partial S}{\partial x_m} \frac{\partial^2 Q_{kl}}{\partial S \partial \varphi} + \frac{\partial \varphi}{\partial x_m}\frac{\partial^2 Q_{kl}}{\partial \varphi^2} \right) \\
		&\qquad\qquad + 2\frac{\partial^2 S}{\partial x_m \partial x_j}\frac{\partial \varphi}{\partial x_j} \frac{\partial^2 Q_{kl}}{\partial \varphi \partial S} + 2\frac{\partial S}{\partial x_j}\frac{\partial^2 \varphi}{\partial x_m \partial x_j}\frac{\partial^2 Q_{kl}}{\partial \varphi \partial S} + 2\frac{\partial S}{\partial x_j}\frac{\partial \varphi}{\partial x_j} \frac{\partial \varphi}{\partial x_m}\frac{\partial^3 Q_{kl}}{\partial \varphi^2 \partial S} \\
		&\qquad\qquad\qquad\qquad + 2\frac{\partial^2 \varphi}{\partial x_m \partial x_j}\frac{\partial \varphi}{\partial x_j} \frac{\partial^2 Q_{kl}}{\partial \varphi^2} + \left( \frac{\partial \varphi}{\partial x_j} \right)^2 \left( \frac{\partial \varphi}{\partial x_m}\frac{\partial^3 Q_{kl}}{\partial \varphi^3} + \frac{\partial S}{\partial x_m}\frac{\partial^3 Q_{kl}}{\partial S\partial \varphi^2} \right)
	\end{split}
	\end{equation}
	Should eventually try to factorize by common $Q_{kl}$ terms so that the sum over $kl$ becomes less painful. The second factor is just two terms by a simple application of the chain rule. Finally, the time derivative term will invoke all the nonvanishing terms from \eqref{eq:expand-begin}-\eqref{eq:expand-end}, except with $x_i \to t$. Thus, there will be a lot of terms for the Biharmonic equation -- I don't want to write them out right now. 
	
	\section*{Outstanding Questions}
	\begin{itemize}
		\item For the three dimensional case, how might we deal with the derivatives in the z-direction. 
		\begin{itemize}
			\item Set all of these fields to be constant in the z-direction so that the corresponding derivatives vanish. Seems physically plausible. Here $\partial_m\partial^2\partial_n\psi_n$ term goes away because $\psi_1 = \psi_2 = 0$ and $\partial_3 \psi_3 = 0$ by our condition on the z-components.
		\end{itemize}
		
		\item How to find initial velocity/stream field?
		\begin{itemize}
			\item One way might be to start with a fixed $Q_{ij}$ field, and then solve the biharmonic equation for $\psi_m$.
		\end{itemize}
	
		\item How to find an initial $Q_{ij}$ field?
		\begin{itemize}
			\item Zumer and Svensek say that they start with their initial configuration corresponding to the two-defect case, and then they evolve it in time ``without hydrodynamics a sufficient number of time steps'' or something like that. It is unclear to me what ``sufficient'' means in this context.
		\end{itemize}
		
		\item Whether/how to update the velocity field upon iteration
		\begin{itemize}
			\item Again, since there's no $\partial v_i/\partial t$ it is not intuitive whether or how to update the velocity field. 
			\item We could solve the biharmonic equation to update it, but that seems to run counter to the idea that $\partial v_i/\partial t = 0$. 
		\end{itemize}
		
		\item If we wanted to keep the equations in terms of $Q_{ij}$ rather than $S$ and $\varphi$, how would we solve for the Lagrange multipliers
		\begin{itemize}
			\item It seems like we treat them as free parameters when we do minimization of the free energy, but here we are updating things based on a discretization of time -- the Lagrange multiplier terms do not have time derivatives. 
			\item Would they be solved for once to find a local minimum of the free energy and then be kept constant?
			\item Do they get updated in time to make sure $Q_{ij}$ is symmetric and traceless at each time step?
		\end{itemize}
		
		\item Is there a smart way of taking these weird combinations of derivatives of $Q_{ij}$ or do we just approximate it in the usual way and hope the error is low?
		
		\item How do we deal with the fact that $\partial \varphi/\partial t$ blows up with $S \to 0$?
		\begin{itemize}
			\item Could we put a limit on how small $S$ can get when we calculate $\partial \varphi/\partial t$ and then denote those regions as defect/isotropic regions? 
			\item If we do that, how would we keep track of the defects moving?
		\end{itemize}
	\end{itemize}
	
\end{document}