\documentclass[reqno]{article}
\usepackage{amsmath}
\usepackage{amssymb}
\usepackage{amsthm}
\usepackage{mathrsfs}
\usepackage{mathtools}
\usepackage{enumerate}
\usepackage{esint}
\usepackage{relsize}
\usepackage{graphicx}
\usepackage{float}
\usepackage{amsthm}
\usepackage{pgf, tikz}
\usepackage{stmaryrd}
\usepackage{hyperref}
\usetikzlibrary{graphs}

\setlength{\textheight}{9truein}
\setlength{\topmargin}{-0.5truein}
\setlength{\textwidth}{6truein}
\setlength{\oddsidemargin}{.25truein}
\setlength{\parskip}{6pt plus 2pt minus 1pt}

\newcommand{\Pic}[1]{\text{Pic}(#1)}
\newcommand{\Div}[1]{\text{Div}(#1)}
\newcommand{\divv}[1]{\text{div}(#1)}
\newcommand{\Z}{\mathbb{Z}}
\newcommand{\Jac}{\text{Jac}}

\newtheorem{lemma}{Lemma}
\newtheorem*{theorem}{Theorem}
\newtheorem*{conjecture}{Theorem}

\setlength{\parindent}{40pt}

\begin{document}
\title{Analytic solution for flow around defect from anisotropic elasticity}
\author{Lucas Myers}
\maketitle

\section{Director field configuration}
The director field configuration is given by:
\begin{equation}
  \mathbf{n} = \left( \cos k\phi, \sin k\phi, 0 \right)
\end{equation}
for 3-dimensions.
Recall also that the $Q$-tensor is given by:
\begin{equation}
  Q_{ij} = \frac{S}{2} \left( 3 n_i n_j - \delta_{ij} \right)
\end{equation}
Plugging this in yields:
\begin{equation}
  \begin{split}
    Q &= \frac{S}{2} \begin{bmatrix}
      \left(3 \cos^{2}k\phi - 1\right)
      & \tfrac32 \sin 2k \phi
      & 0 \\
      \tfrac32 \sin 2k \phi
      & \left(3 \sin^{2}k\phi - 1\right)
      & 0 \\
      0
      & 0
      & -1
    \end{bmatrix} \\
    &= \frac{S}{2} \begin{bmatrix}
      \left(\tfrac12 + \tfrac32 \cos 2k\phi\right)
      & \tfrac32 \sin 2k \phi
      & 0 \\
      \tfrac32 \sin 2k \phi
      & \left(\tfrac12 - \tfrac32 \cos 2k\phi \right)
      & 0 \\
      0
      & 0
      & -1
    \end{bmatrix}
  \end{split}
\end{equation}

Now, the flow equation for an equilibrium configuration is given by:
\begin{equation}
	\partial^4 \psi = \frac{1}{b} \Bigl[
	2 \bigl( \Phi_{L_1} (Q) 
	+ \kappa_2 \Phi_{L_2} (Q)
	+ \kappa_3 \Phi_{L_3} (Q) \bigr)
	+ a \Phi_{\mu_2} (Q)
	\Bigr]
\end{equation}
with
\begin{equation}
	\Phi_{L_1} (Q) \equiv -\epsilon_{ki} \left( \partial_k \partial_j^2 Q_{mn} \right)
	\left( \partial_i Q_{mn} \right)
\end{equation}
and
\begin{equation}
	\Phi_{L_3}(Q)
	= 
	\begin{multlined}[t]
	-2 L_3 \epsilon_{npi} \bigl[ 
	\left( \partial_p \partial_j Q_{jm} \right) \left( \partial_m Q_{kl} \right) \left( \partial_i Q_{kl} \right) \\
	\hspace{1.5cm}+ \left( \partial_j Q_{jm} \right) \left( \partial_p \partial_m Q_{kl} \right) \left( \partial_i Q_{kl} \right) \\
	\hspace{1.5cm}+ \left( \partial_p Q_{jm} \right) \left( \partial_j \partial_m Q_{kl} \right) \left( \partial_i Q_{kl} \right) \\
	\hspace{1.5cm}+ Q_{jm} \left( \partial_p \partial_j \partial_m Q_{kl} \right) \left( \partial_i Q_{kl} \right) \\
	\hspace{1.5cm}+ \left( \partial_p Q_{jm} \right) \left( \partial_m Q_{kl} \right) \left( \partial_i \partial_j Q_{kl} \right)
	\bigr]
	\end{multlined}
\end{equation}
This equation is derived in the ``FlowFromElasticForces'' file.
Note that we are unconcerned with the $\Phi_{L_2}$ and $\Phi_{\mu_2}$ terms, because for this scenario we set $\kappa_2 = 0$ and assume an equilibrium configuration.

If we plug in the $Q$-configuration from above, this yields the following:
\begin{equation}
  \Phi_{L_1}\left(Q\right) = 0
\end{equation}
Clearly a solution to the Biharmonic equation is $\psi = 0$.
See \href{https://math.stackexchange.com/q/156366/683123}{this stack exchange post} for an explanation of uniqueness of this solution.

For the anisotropic term, we get that:
\begin{equation}
  \Phi_{L_3}(Q) =
  \frac{27 S^{3}}{256 r^{4}} \begin{multlined}[t]
    \bigl[
    64 \sin^{6}\left(\frac{\phi}{2} \right) \sin\left(2 \phi \right)
    + 96 \sin^{4}\left(\frac{\phi}{2} \right) \sin\left(\phi \right) \\
    - 96 \sin^{4}\left(\frac{\phi}{2} \right) \sin\left(2 \phi \right)
    - 64 \sin^{4}\left(\frac{\phi}{2} \right) \sin\left(3 \phi \right) \\
    - 12 \sin\left(\phi \right) 
    + 24 \sin\left(2 \phi \right) 
    + 9 \sin\left(3 \phi \right)
    - 16 \sin\left(4 \phi \right)
    + 5 \sin\left(5 \phi \right)
    \bigr]
    \end{multlined}
\end{equation}
If we simplify this using various trig identities, we're left with the very simple expression:
\begin{equation}
  \Phi_{L_3} (Q) = \frac{81 S^{3} \sin\left(\phi \right)}{32 r^{4}}
\end{equation}

\end{document}
