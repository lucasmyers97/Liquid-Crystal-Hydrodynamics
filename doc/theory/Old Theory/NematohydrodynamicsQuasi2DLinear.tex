\documentclass[reqno]{article}
\usepackage{amsmath}
\usepackage{amssymb}
\usepackage{amsthm}
\usepackage{mathrsfs}
\usepackage{enumerate}
\usepackage{esint}
\usepackage{relsize}
\usepackage{graphicx}
\usepackage{float}
\usepackage{amsthm}
\usepackage{tikz}
\usepackage{stmaryrd}
\usetikzlibrary{graphs}

\setlength{\textheight}{9truein}
\setlength{\topmargin}{-0.5truein}
\setlength{\textwidth}{6truein}
\setlength{\oddsidemargin}{.25truein}
\setlength{\parskip}{6pt plus 2pt minus 1pt}

\newcommand{\Pic}[1]{\text{Pic}(#1)}
\newcommand{\Div}[1]{\text{Div}(#1)}
\newcommand{\divv}[1]{\text{div}(#1)}
\newcommand{\Z}{\mathbb{Z}}
\newcommand{\Jac}{\text{Jac}}

\newtheorem{lemma}{Lemma}
\newtheorem*{theorem}{Theorem}
\newtheorem*{conjecture}{Theorem}

\setlength{\parindent}{40pt}

\begin{document}
	\title{Nematohydrodynamics, Quasi-2D and Linear Approximations}
	\author{Lucas Myers}
	\maketitle
	
	\section*{Assumptions}
	We will consider a coupled nematic liquid crystal and hydrodynamic system on a flat substrate. We assume the fluid velocity and Q-tensor magnitude are small so we neglect terms higher than order-1 in $Q_{ij}$ and $v_i$. Additionally, we assume no acceleration so that $\partial v_i/\partial t = 0$ always. We give an initial director field configuration of $\mathbf{n} = (\cos\phi, \sin\phi, 0)$ for some $\varphi$ (to be specified) so that the Q-tensor, given by $Q_{ij} = S/2(3n_i n_j - \delta_{ij})$ takes the form:
	\begin{align}
		Q_{ij} &= \frac{S}{2}\left[
			\begin{matrix}
				3\cos^2\varphi - 1 & 3\cos\varphi\sin\varphi & 0 \\
				3\cos\varphi\sin\varphi & 3\sin^2\varphi - 1 & 0 \\
				0 & 0 & -1
			\end{matrix}
		\right]\\
		&= \frac{S}{2}\left[
			\begin{matrix}
				3\cos^2\varphi - 1 & \frac{3}{2}\sin2\varphi & 0 \\
				\frac{3}{2}\sin2\varphi & 3\sin^2\varphi - 1 & 0 \\
				0 & 0 & -1
			\end{matrix}
		\right]
	\end{align}
	
	\section*{Computing homoegenous and elastic generalized force}
	From Svensek and Zumer, the free energy density is given by
	\begin{equation}
		f = \phi(Q) + \frac{1}{2} L \partial_i Q_{jk} \partial_i Q_{jk}
	\end{equation}
	so that the homogeneous elastic part of the generalized force is given by:
	\begin{equation}
		h^{he}_{ij} = L\partial_k^2 Q_{ij} - \frac{\partial \phi}{\partial Q_{ij}} + \lambda \delta_{ij} + \lambda_k\epsilon_{kij}
	\end{equation}
	where $\phi$ is the Landau de Gennes bulk free energy:
	\begin{equation}
		\phi(Q) = \frac{1}{2}A Q_{ij}Q_{ji} + \frac{1}{3}B Q_{ij}Q_{jk}Q_{ki} + \frac{1}{4} C(Q_{ij}Q_{ji})^2
	\end{equation}
	We begin by explicitly calculating the second term in terms of $Q_{ij}$, one term at a time in $\phi$:
	\begin{equation}
	\begin{split}
		\frac{\partial}{\partial Q_{mn}}\left(\frac{1}{2}AQ_{ij}Q_{ji}\right) &= \frac{1}{2}A \left(\delta_{im}\delta_{jn}Q_{ji} + Q_{ij}\delta_{jm}\delta_{in}\right) \\
		&= \frac{1}{2}A (Q_{nm} + Q_{nm}) \\
		&= A Q_{mn}
	\end{split}
	\end{equation}
	where in the last step we've used symmetry of $Q_{ij}$.
	\begin{equation}
	\begin{split}
		\frac{\partial}{\partial Q_{mn}} \left( \frac{1}{3}BQ_{ij}Q_{jk}Q_{ki} \right) &= \frac{1}{3}B ( \delta_{im}\delta_{jn}Q_{jk}Q_{ki} + Q_{ij}\delta_{jm}\delta_{kn}Q_{ki} + Q_{ij}Q_{jk}\delta_{mk}\delta_{ni} ) \\
		&= \frac{1}{3}B(Q_{nk}Q_{km} + Q_{im}Q_{ni} + Q_{nj}Q_{jm}) \\
		&= BQ_{ni}Q_{im}
	\end{split}
	\end{equation}
	And the final term gives:
	\begin{equation}
	\begin{split}
		\frac{\partial}{\partial Q_{mn}}\left( \frac{1}{4} C(Q_{ij}Q_{ji})^2 \right) &= \frac{1}{4}C \cdot 2(Q_{ij}Q_{ji}) \frac{\partial(Q_{kl}Q_{lk})}{\partial Q_{mn}} \\
		&= \frac{1}{4}C \cdot 2(Q_{ij}Q_{ji}) \cdot (\delta_{mk}\delta_{nl}Q_{lk} + Q_{kl}\delta_{lm}\delta_{kn}) \\
		&= \frac{1}{4}C \cdot 2(Q_{ij}Q_{ji}) \cdot (Q_{nm} + Q_{nm}) \\
		&= C Q_{mn} (Q_{ij}Q_{ji})
	\end{split}
	\end{equation}
	Thus, the total homogeneous and elastic force reads:
	\begin{equation}
		h^{he}_{ij} = L\partial^2 Q_{ij} - A Q_{mn} - BQ_{ni}Q_{im} - C Q_{mn} (Q_{ij}Q_{ji}) + \lambda\delta_{ij} + \lambda_k \epsilon_{kij}
	\end{equation}
	
	\section*{Computing viscous force explicitly}
	Alrighty then, now we need the viscous force on the liquid crystals. From Svensek and Zumer, the viscous force is given by:
	\begin{equation}
		-h^v_{ij} = \frac{1}{2} \mu_2 A_{ij} + \mu_1 N_{ij}
	\end{equation}
	with 
	\begin{equation}
		N_{ij} = \frac{d Q_{ij}}{dt} + W_{ik} Q_{kj} - Q_{ik} W_{kj}
	\end{equation}
	and
	\begin{equation}
		\frac{d Q_{ij}}{dt} = \frac{\partial Q_{ij}}{\partial t} + (v\cdot \nabla)Q_{ij}
	\end{equation}
	The second two terms in the expression for $N_{ij}$ are quadratic in $v_i$ and $Q_{ij}$ so we may drop them, and $(v\cdot \nabla)Q_{ij}$ is clearly quadratic. Hence, we make the approximation
	\begin{equation}
		N_{ij} \approx \frac{\partial Q_{ij}}{\partial t}
	\end{equation}
	We also have the definition
	\begin{equation}
		A_{ij} = (\partial_i v_j + \partial_j v_i)
	\end{equation}
	Note that we want to restrict our analysis to an incompressible fluid, so we must have a further restriction that:
	\begin{equation}
		\partial_i v_i = 0
	\end{equation}
	Then the expression for the complete viscous force is
	\begin{equation}
		-h^v_{ij} = \frac{1}{2}\mu_2 (\partial_i v_j + \partial_j v_i) + \mu_1 \frac{\partial Q_{ij}}{\partial t}
	\end{equation}
	One equation of motion is then given by 
	\begin{equation}
		h^{he}_{ij} + h^{v}_{ij} = 0
	\end{equation}
	or explicitly
	\begin{equation}
		\mu_1 \frac{\partial Q_{ij}}{\partial t} = L\partial^2 Q_{ij} - A Q_{mn} - BQ_{ni}Q_{im} - C Q_{mn} (Q_{ij}Q_{ji}) + \lambda\delta_{ij} + \lambda_k \epsilon_{kij} - \frac{1}{2}\mu_2 (\partial_i v_j + \partial_j v_i)
	\end{equation}
	Using this, we may update $Q_{ij}$ in time by solving for $\partial Q_{ij}/\partial t$ in terms of $Q_{ij}$ and $v_i$ from the previous iteration.
	
	\section*{Computing the elastic stress tensor explicitly}
	The elastic stress tensor is obtained via
	\begin{equation}
		\sigma^{e}_{ij} = -\frac{\partial f}{\partial (\partial_i Q_{kl})} \partial_j Q_{kl}
	\end{equation}
	Note that only the elastic part of the free energy make references to derivatives:
	\begin{equation}
	\begin{split}
		\frac{\partial f}{\partial (\partial_i Q_{kl})} &= \frac{\partial}{\partial (\partial_i Q_{kl})} \frac{1}{2} L \partial_j Q_{mn} \partial_j Q_{mn} \\
		&= \frac{1}{2} L (\delta_{ij}\delta_{km}\delta_{ln} \partial_j Q_{mn} + \partial_j Q_{mn} \delta_{ij}\delta_{km}\delta_{ln}) \\
		&= \frac{1}{2} L (\partial_i Q_{kl} + \partial_i Q_{kl}) \\
		&= L\partial_i Q_{kl}
	\end{split}
	\end{equation}
	Then the elastic stress tensor is given by
	\begin{equation}
		\sigma^e_{ij} = -L\partial_i Q_{kl} \partial_j Q_{kl}
	\end{equation}
	
	\section*{Computing viscous stress tensor explicitly}
	The viscous stress tensor is given by
	\begin{equation}
		\sigma^v_{ij} = \beta_1 Q_{ij}Q_{kl}A_{kl} + \beta_4 A_{ij} + \beta_5 Q_{ik} A_{ki} + \frac{1}{2} \mu_2 N_{ij} - \mu_1 Q_{ik}N_{kj} + \mu_1 Q_{jk}N_{ki}
	\end{equation}
	However, only the $\beta_4$ and $\mu_2$ are linear in $Q_{ij}$ and $v_i$. Hence, this simplifies to
	\begin{equation}
		\sigma^v_{ij} \approx \beta_4 A_{ij} + \frac{1}{2} \mu_2 N_{ij}
	\end{equation}
	Again, plugging in for $A_{ij}$ and $N_{ij}$ as we did for the viscous force, we get
	\begin{equation}
		\sigma^v_{ij} \approx \beta_4 (\partial_i v_j + \partial_j v_i) + \frac{1}{2} \mu_2  \frac{\partial Q_{ij}}{\partial t}
	\end{equation}
	\section*{Computing the fluid equation of motion}
	The equation of motion for the fluid is given by
	\begin{equation}
		\rho \frac{\partial v_i}{\partial t} = -\partial_i p + \partial_j (\sigma^v_{ji} + \sigma^{e}_{ji})
	\end{equation}
	We've made the assumption that $\partial v_i/\partial t \approx 0$. Plugging in for $\sigma^v_{ji}$ and $\sigma^e_{ji}$ yields
	\begin{equation}
	\begin{split}
		0 &= -\partial_i p + \partial_j \left( -L\partial_j Q_{kl} \partial_i Q_{kl} + \beta_4 (\partial_j v_i + \partial_i v_j) + \frac{1}{2} \mu_2  \frac{\partial Q_{ji}}{\partial t} \right) \\
		&= -\partial_i p - L\left[(\partial^2 Q_{kl}) \partial_i Q_{kl} + (\partial_j Q_{kl}) \partial_j \partial_i Q_{kl}\right] + \beta_4(\partial^2 v_i + \partial_j \partial_i v_j) + \frac{1}{2} \mu_2 \partial_j \frac{\partial Q_{ji}}{\partial t} \\
		&= -\partial_i p - L\left[(\partial^2 Q_{kl}) \partial_i Q_{kl} + (\partial_j Q_{kl}) \partial_j \partial_i Q_{kl}\right] + \beta_4\partial^2 v_i + \frac{1}{2} \mu_2 \partial_j \frac{\partial Q_{ji}}{\partial t}
	\end{split}
	\end{equation}
	where in the last step we have used the condition that $\partial_j v_j = 0$. 
	
\end{document}