\documentclass[leqno]{article}
\usepackage{amsmath}
\usepackage{amssymb}
\usepackage{amsthm}
\usepackage{mathrsfs}
\usepackage{enumerate}
\usepackage{esint}
\usepackage{relsize}
\usepackage{graphicx}
\usepackage{float}
\usepackage{amsthm}
\usepackage{tikz}
\usepackage{stmaryrd}
\usetikzlibrary{graphs}

\setlength{\textheight}{9truein}
\setlength{\topmargin}{-0.5truein}
\setlength{\textwidth}{6truein}
\setlength{\oddsidemargin}{.25truein}
\setlength{\parskip}{6pt plus 2pt minus 1pt}

\newcommand{\Pic}[1]{\text{Pic}(#1)}
\newcommand{\Div}[1]{\text{Div}(#1)}
\newcommand{\divv}[1]{\text{div}(#1)}
\newcommand{\Z}{\mathbb{Z}}
\newcommand{\Jac}{\text{Jac}}

\newtheorem{lemma}{Lemma}
\newtheorem*{theorem}{Theorem}
\newtheorem*{conjecture}{Theorem}

\setlength{\parindent}{40pt}

\begin{document}
	\title{Nematohydrodynamics, Quasi-2D and Linear Approximations}
	\author{Lucas Myers}
	\maketitle
	
	\section*{Assumptions}
	We will consider a coupled nematic liquid crystal and hydrodynamic system on a flat substrate. Given this, we will assume a quasi-2D system -- that is, the nematic crystal director field will only lie in the x-y plane. Additionally, we assume the fluid velocity and Q-tensor magnitude are small so we neglect terms higher than order-1 in $Q_{ij}$ and $v_i$. Finally, we assume no acceleration so that $\partial v_i/\partial t = 0$ always. From the first assumption, we may assume the director field takes the form $\mathbf{n} = (\cos\phi, \sin\phi, 0)$ so that the Q-tensor, given by $Q_{ij} = S/2(3n_i n_j - \delta_{ij})$ takes the form:
	\begin{align*}
		Q_{ij} &= \frac{S}{2}\left[
			\begin{matrix}
				3\cos^2\varphi - 1 & 3\cos\varphi\sin\varphi & 0 \\
				3\cos\varphi\sin\varphi & 3\sin^2\varphi - 1 & 0 \\
				0 & 0 & -1
			\end{matrix}
		\right]\\
		&= \frac{S}{2}\left[
			\begin{matrix}
				3\cos^2\varphi - 1 & \frac{3}{2}\sin2\varphi & 0 \\
				\frac{3}{2}\sin2\varphi & 3\sin^2\varphi - 1 & 0 \\
				0 & 0 & -1
			\end{matrix}
		\right]
	\end{align*}
	
	\section*{Computing homoegenous and elastic generalized force in terms of $S$ and $\phi$}
	From Svensek and Zumer, the free energy density is given by
	\[
		f = \phi(Q) + \frac{1}{2} L \partial_i Q_{jk} \partial_i Q_{jk}
	\]
	so that the homogeneous elastic part of the generalized force is given by:
	\[
		h^{he}_{ij} = L\partial_k^2 Q_{ij} - \frac{\partial \phi}{\partial Q_{ij}} + \lambda \delta_{ij} + \lambda_k\epsilon{kij}
	\]
	where $\phi$ is the Landau de Gennes bulk free energy:
	\[
		\phi(Q) = \frac{1}{2}A Q_{ij}Q_{ji} + \frac{1}{3}B Q_{ij}Q_{jk}Q_{ki} + \frac{1}{4} C(Q_{ij}Q_{ji})^2
	\]
	Note that, by our assumptions on $Q_{ij}$ it will always be traceless and symmetric, so we may omit the Lagrange multiplier terms. We begin by explicitly calculating the second term in terms of $Q_{ij}$, one term at a time in $\phi$:
	\begin{align*}
		\frac{\partial}{\partial Q_{mn}}\left(\frac{1}{2}AQ_{ij}Q_{ji}\right) &= \frac{1}{2}A \left(\delta_{im}\delta_{jn}Q_{ji} + Q_{ij}\delta_{jm}\delta_{in}\right) \\
		&= \frac{1}{2}A (Q_{nm} + Q_{nm}) \\
		&= A Q_{mn}
	\end{align*}
	where in the last step we've used symmetry of $Q_{ij}$.
	\begin{align*}
		\frac{\partial}{\partial Q_{mn}} \left( \frac{1}{3}BQ_{ij}Q_{jk}Q_{ki} \right) &= \frac{1}{3}B ( \delta_{im}\delta_{jn}Q_{jk}Q_{ki} + Q_{ij}\delta_{jm}\delta_{kn}Q_{ki} + Q_{ij}Q_{jk}\delta_{mk}\delta_{ni} ) \\
		&= \frac{1}{3}B(Q_{nk}Q_{km} + Q_{im}Q_{ni} + Q_{nj}Q_{jm}) \\
		&= BQ_{ni}Q_{im}
	\end{align*}
	And the final term gives:
	\begin{align*}
		\frac{\partial}{\partial Q_{mn}}\left( \frac{1}{4} C(Q_{ij}Q_{ji})^2 \right) &= \frac{1}{4}C \cdot 2(Q_{ij}Q_{ji}) \frac{\partial(Q_{kl}Q_{lk})}{\partial Q_{mn}} \\
		&= \frac{1}{4}C \cdot 2(Q_{ij}Q_{ji}) \cdot (\delta_{mk}\delta_{nl}Q_{lk} + Q_{kl}\delta_{lm}\delta_{kn}) \\
		&= \frac{1}{4}C \cdot 2(Q_{ij}Q_{ji}) \cdot (Q_{nm} + Q_{nm}) \\
		&= C Q_{mn} (Q_{ij}Q_{ji})
	\end{align*}
	We may explicitly calculate the matrices corresponding to these contractions later. Now we calculate the isotropic elastic term. Note that $Q$ is a function of $S$ and $\varphi$, and we would like to write the spatial derivatives in terms of those two parameters. 
	\begin{align*}
		L\partial_k^2 Q_{ij} &= L\partial_k \left[\frac{\partial S}{\partial x_k} \frac{\partial Q_{ij}}{\partial S} + \frac{\partial \varphi}{\partial x_k} \frac{\partial Q_{ij}}{\partial \varphi} \right]\\
		&= L \left[ \frac{\partial^2 S}{\partial x_k^2} \frac{\partial Q_{ij}}{\partial S} + \frac{\partial S}{\partial x_k} \frac{\partial}{\partial x_k} \left( \frac{\partial Q_{ij}}{\partial S}\right) +  \frac{\partial^2 \varphi}{\partial x_k^2} \frac{\partial Q_{ij}}{\partial \varphi} + \frac{\partial \varphi}{\partial x_k} \frac{\partial}{\partial x_k} \left( \frac{\partial Q_{ij}}{\partial \varphi} \right) \right] \\
		&= L\left[ \frac{\partial^2 S}{\partial x_k^2} \frac{\partial Q_{ij}}{\partial S} + \frac{\partial^2 \varphi}{\partial x_k^2} \frac{\partial Q_{ij}}{\partial \varphi} + \frac{\partial S}{\partial x_k} \left( \frac{\partial S}{\partial x_k} \frac{\partial^2 Q_{ij}}{\partial S^2} + \frac{\partial \varphi}{\partial x_k} \frac{\partial^2 Q_{ij}}{\partial \varphi \partial S} \right) + \frac{\partial \varphi}{\partial x_k} \left( \frac{\partial \varphi}{\partial x_k} \frac{\partial^2 Q_{ij}}{\partial \varphi^2} + \frac{\partial S}{\partial x_k} \frac{\partial^2 Q_{ij}}{\partial S \partial \varphi} \right) \right] \\
		&= L\left[ \frac{\partial^2 S}{\partial x_k^2} \frac{\partial Q_{ij}}{\partial S} + \frac{\partial^2 \varphi}{\partial x_k^2} \frac{\partial Q_{ij}}{\partial \varphi} + 2 \frac{\partial S}{\partial x_k} \frac{\partial \varphi}{\partial x_k} \frac{\partial^2 Q_{ij}}{\partial S \partial \varphi} + \left( \frac{\partial S}{\partial x_k} \right)^2 \frac{\partial^2 Q_{ij}}{\partial S^2} + \left( \frac{\partial \varphi}{\partial x_k} \right)^2 \frac{\partial^2 Q_{ij}}{\partial \varphi^2} \right]
	\end{align*}
	Thus, the total homogeneous and elastic force reads:
	\begin{multline*}
		h^{he}_{ij} = L\left[ \frac{\partial^2 S}{\partial x_k^2} \frac{\partial Q_{ij}}{\partial S} + \frac{\partial^2 \varphi}{\partial x_k^2} \frac{\partial Q_{ij}}{\partial \varphi} + 2 \frac{\partial S}{\partial x_k} \frac{\partial \varphi}{\partial x_k} \frac{\partial^2 Q_{ij}}{\partial S \partial \varphi} + \left( \frac{\partial S}{\partial x_k} \right)^2 \frac{\partial^2 Q_{ij}}{\partial S^2} + \left( \frac{\partial \varphi}{\partial x_k} \right)^2 \frac{\partial^2 Q_{ij}}{\partial \varphi^2} \right] \\ - A Q_{mn} - BQ_{ni}Q_{im} - C Q_{mn} (Q_{ij}Q_{ji})
	\end{multline*}
	
	\section*{Computing viscous force in terms of $S$, $\phi$ and $\psi_i$}
	Alrighty then, now we need the viscous force on the liquid crystals. From Svensek and Zumer, the viscous force is given by:
	\[
		-h^v_{ij} = \frac{1}{2} \mu_2 A_{ij} + \mu_1 N_{ij}
	\]
	with 
	\[
		N_{ij} = \frac{d Q_{ij}}{dt} + W_{ik} Q_{kj} - Q_{ik} W_{kj}
	\]
	and
	\[
		\frac{d Q_{ij}}{dt} = \frac{\partial Q_{ij}}{\partial t} + (v\cdot \nabla)Q_{ij}
	\]
	The second two terms in the expression for $N_{ij}$ are quadratic in $v_i$ and $Q_{ij}$ so we may drop them, and $(v\cdot \nabla)Q_{ij}$ is clearly quadratic. Hence, we make the approximation
	\[
		N_{ij} \approx \frac{\partial Q_{ij}}{\partial t}
	\]
	We also have the definition
	\[
		A_{ij} = (\partial_i v_j + \partial_j v_i)
	\]
	Note that we want to restrict our analysis to an incompressible fluid, so we must have the restriction that
	\[
		\partial_i v_i = 0
	\]
	It turns out that, on any simply-connected domain (I think we will be using a square here so we're good), one may express such a vector field as the curl of another vector field. Hence we define $\psi_k$ by
	\[
		v_i = \epsilon_{ijk} \partial_j \psi_k
	\]
	Hence, we may plug this into our expression for $A_{ij}$ to get
	\begin{align*}
		A_{ij} &= (\partial_i v_j + \partial_j v_i) \\
		&= \partial_i \epsilon_{jkl} \partial_k \psi_l + \partial_j \epsilon_{ikl} \partial_k \psi_l \\
		&= (\epsilon_{ikl} \partial_j + \epsilon_{jkl} \partial_i)\partial_k \psi_l
	\end{align*}
	And then for $N_{ij}$ we have
	\begin{align*}
		N_{ij} &\approx \frac{\partial Q_{ij}}{\partial t} \\
		&= \frac{\partial S}{\partial t} \frac{\partial Q_{ij}}{\partial S} + \frac{\partial \varphi}{\partial t} \frac{\partial Q_{ij}}{\partial \varphi}
	\end{align*}
	Then the expression for the complete viscous force is
	\[
		-h^v_{ij} = \frac{1}{2}\mu_2 (\epsilon_{ikl} \partial_j + \epsilon_{jkl} \partial_i)\partial_k \psi_l + \mu_1 \frac{\partial S}{\partial t} \frac{\partial Q_{ij}}{\partial S} + \frac{\partial \varphi}{\partial t} \frac{\partial Q_{ij}}{\partial \varphi}
	\]
	One equation of motion is then given by 
	\[
		h^{he}_{ij} + h^{v}_{ij} = 0
	\]
	
	\section*{Computing the elastic stress tensor explicitly}
	The elastic stress tensor is obtained via
	\[
		\sigma^{e}_{ij} = -\frac{\partial f}{\partial (\partial_i Q_{kl})} \partial_j Q_{kl}
	\]
	Note that only the elastic part of the free energy make references to derivatives:
	\begin{align*}
		\frac{\partial f}{\partial (\partial_i Q_{kl})} &= \frac{\partial}{\partial (\partial_i Q_{kl})} \frac{1}{2} L \partial_j Q_{mn} \partial_j Q_{mn} \\
		&= \frac{1}{2} L (\delta_{ij}\delta_{km}\delta_{ln} \partial_j Q_{mn} + \partial_j Q_{mn} \delta_{ij}\delta_{km}\delta_{ln}) \\
		&= \frac{1}{2} L (\partial_i Q_{kl} + \partial_i Q_{kl}) \\
		&= L\partial_i Q_{kl}
	\end{align*}
	Then the elastic stress tensor is given by
	\[
		\sigma^e_{ij} = -L\partial_i Q_{kl} \partial_j Q_{kl}
	\]
	Using the chain rule to find this as a function of $S$ and $\varphi$
	\begin{align*}
		\sigma^e_{ij} &= -L \left( \frac{\partial S}{\partial x_i} \frac{\partial Q_{kl}}{\partial S} + \frac{\partial \varphi}{\partial x_i} \frac{\partial Q_{kl}}{\partial \varphi} \right) \left( \frac{\partial S}{\partial x_j} \frac{\partial Q_{kl}}{\partial S} + \frac{\partial \varphi}{\partial x_j} \frac{\partial Q_{kl}}{\partial \varphi} \right) \\
		&= -L \left[ \left( \frac{\partial Q_{kl}}{\partial S} \right)^2 \frac{\partial S}{\partial x_i}\frac{\partial S}{\partial x_j} + \left( \frac{\partial Q_{kl}}{\partial \varphi} \right)^2 \frac{\partial \varphi}{\partial x_i}\frac{\partial \varphi}{\partial x_j} + \frac{\partial Q_{kl}}{\partial S} \frac{\partial Q_{kl}}{\partial \varphi}\left( \frac{\partial S}{\partial x_i} \frac{\partial \varphi}{\partial x_j} + \frac{\partial S}{\partial x_j} \frac{\partial \varphi}{\partial x_i} \right) \right]
	\end{align*}
	
	\section*{Computing viscous stress tensor explicitly}
	The viscous stress tensor is given by
	\[
		\sigma^v_{ij} = \beta_1 Q_{ij}Q_{kl}A_{kl} + \beta_4 A_{ij} + \beta_5 Q_{ik} A_{ki} + \frac{1}{2} \mu_2 N_{ij} - \mu_1 Q_{ik}N_{kj} + \mu_1 Q_{jk}N_{ki}
	\]
	However, only the $\beta_4$ and $\mu_2$ are linear in $Q_{ij}$ and $v_i$. Hence, this simplifies to
	\[
		\sigma^v_{ij} \approx \beta_4 A_{ij} + \frac{1}{2} \mu_2 N_{ij}
	\]
	Again, plugging in for $A_{ij}$ and $N_{ij}$ as we did for the viscous force, we get
	\[
		\sigma^v_{ij} \approx \beta_4 (\epsilon_{ikl} \partial_j + \epsilon_{jkl} \partial_i)\partial_k \psi_l + \frac{1}{2} \mu_2  \left( \frac{\partial S}{\partial t} \frac{\partial Q_{ij}}{\partial S} + \frac{\partial \varphi}{\partial t} \frac{\partial Q_{ij}}{\partial \varphi} \right)
	\]
	\section*{Computing the fluid equation of motion}
	The equation of motion for the fluid is given by
	\[
		\rho \frac{\partial v_i}{\partial t} = -\partial_i p + \partial_j (\sigma^v_{ji} + \sigma^{e}_{ji})
	\]
	We've made the assumption that $\partial v_i/\partial t \approx 0$. Additionally, we would like to get rid of the pressure term. For this, we take the curl of the whole expression to yield:
	\[
		0 = -\epsilon_{ijk} \partial_j \partial_k p + \epsilon_{ijk} \partial_j \partial_l (\sigma^v_{lk} + \sigma^e_{lk})
	\]
	The first term is zero by antisymmetry of the Levi-Civita tensor and commutativity of partial derivatives. Hence, we're left with
	\[
		\epsilon_{ijk} \partial_j \partial_l (\sigma^v_{lk} + \sigma^e_{lk}) = 0
	\]
	Here things get a little out of hand, just because we end up with a ton of derivatives. In any case, we continue to calculate term by term:
	\begin{align*}
		\partial_l \sigma^e_{lk} = -L\partial_l\left[ \frac{\partial Q_{mn}}{\partial S} \frac{\partial Q_{mn}}{\partial S} \frac{\partial S}{\partial x_l} \frac{\partial S}{\partial x_k} + \frac{\partial Q_{mn}}{\partial \varphi} \frac{\partial Q_{mn}}{\partial \varphi} \frac{\partial \varphi}{\partial x_l} \frac{\partial \varphi}{\partial x_k} + \frac{\partial Q_{mn}}{\partial S} \frac{\partial Q_{mn}}{\varphi} \left( \frac{\partial S}{\partial x_l} \frac{\partial \varphi}{\partial x_k} + \frac{\partial S}{\partial x_k}\frac{\partial \varphi}{\partial x_l} \right) \right]
	\end{align*}
	First term first:
	\begin{align*}
		\partial_l \left( \frac{\partial Q_{mn}}{\partial S} \frac{\partial Q_{mn}}{\partial S} \frac{\partial S}{\partial x_l} \frac{\partial S}{\partial x_k} \right) 
		&= \left( \frac{\partial S}{\partial x_l} \frac{\partial^2 Q_{mn}}{\partial S^2} + \frac{\partial \varphi}{\partial x_l} \frac{\partial^2 Q_{mn}}{\partial \varphi \partial S} \right)\frac{\partial Q_{mn}}{\partial S} \frac{\partial S}{\partial x_l} \frac{\partial S}{\partial x_k} \\
		&\qquad+ \frac{\partial Q_{mn}}{\partial S} \left( \frac{\partial S}{\partial x_l} \frac{\partial^2 Q_{mn}}{\partial S^2} + \frac{\partial \varphi}{\partial x_l} \frac{\partial^2 Q_{mn}}{\partial \varphi \partial S} \right)\frac{\partial S}{\partial x_l} \frac{\partial S}{\partial x_k} \\
		&\qquad\qquad+ \frac{\partial Q_{mn}}{\partial S} \frac{\partial Q_{mn}}{\partial S} \left( \frac{\partial^2 S}{\partial x_l^2} \right) \frac{\partial S}{\partial x_k} \\
		&\qquad\qquad\qquad+ \frac{\partial Q_{mn}}{\partial S} \frac{\partial Q_{mn}}{\partial S} \frac{\partial S}{\partial x_l} \left( \frac{\partial^2 S}{\partial x_l \partial x_k} \right) \\
		&= 2\left( \frac{\partial S}{\partial x_l} \frac{\partial^2 Q_{mn}}{\partial S^2} + \frac{\partial \varphi}{\partial x_l} \frac{\partial^2 Q_{mn}}{\partial \varphi \partial S} \right)\frac{\partial Q_{mn}}{\partial S} \frac{\partial S}{\partial x_l} \frac{\partial S}{\partial x_k} \\
		&\qquad+ \frac{\partial Q_{mn}}{\partial S} \frac{\partial Q_{mn}}{\partial S} \left( \frac{\partial^2 S}{\partial x_l^2} \frac{\partial S}{\partial x_k} + \frac{\partial S}{\partial x_l} \frac{\partial^2 S}{\partial x_l \partial x_k} \right) \\
	\end{align*}
	Now for the $x_j$ derivative (one term at a time):
	\begin{align*}
		\partial_k 2 \frac{\partial^2 Q_{mn}}{\partial \varphi \partial S} \frac{\partial Q_{mn}}{\partial S} \frac{\partial S}{\partial x_l} \frac{\partial S}{\partial x_k} \frac{\partial \varphi}{\partial x_l} 
		&= 2 \left( \frac{\partial S}{\partial x_k} \frac{\partial^3 Q_{mn}}{\partial \varphi \partial S^2} + \frac{\partial \varphi}{\partial x_k} \frac{\partial^3 Q_{mn}}{\partial S \partial \varphi^2} \right) \frac{\partial Q_{mn}}{\partial S} \frac{\partial S}{\partial x_l} \frac{\partial S}{\partial x_k} \frac{\partial \varphi}{\partial x_l} \\
		&+\qquad 2 \frac{\partial^2 Q_{mn}}{\partial \varphi \partial S} \left( \frac{\partial S}{\partial x_k} \frac{\partial^2 Q_{mn}}{\partial } \right)
	\end{align*}
	
\end{document}