\documentclass[reqno]{article}
\usepackage{amsmath}
\usepackage{amssymb}
\usepackage{amsthm}
\usepackage{mathrsfs}
\usepackage{mathtools}
\usepackage{enumerate}
\usepackage{esint}
\usepackage{relsize}
\usepackage{graphicx}
\usepackage{float}
\usepackage{amsthm}
\usepackage{pgf, tikz}
\usepackage{stmaryrd}
\usepackage{hyperref}
\usetikzlibrary{graphs}

\setlength{\textheight}{9truein}
\setlength{\topmargin}{-0.5truein}
\setlength{\textwidth}{6truein}
\setlength{\oddsidemargin}{.25truein}
\setlength{\parskip}{6pt plus 2pt minus 1pt}

\setlength{\parindent}{40pt}

\begin{document}
	\title{LC Flow Velocity Eqs}
	\author{Lucas Myers}
	\maketitle
	
	\section{Elastic generalized force}
	From Svensek and Zumer, the free energy density is given by
	\begin{equation}
		f = \phi(Q) + \frac{1}{2} L \partial_i Q_{jk} \partial_i Q_{jk}
	\end{equation}
	so that the homogeneous elastic part of the generalized force is given by:
	\begin{equation}
		h^{he}_{ij} 
		= L\partial_k^2 Q_{ij} 
		- \frac{\partial \phi}{\partial Q_{ij}} 
		+ \lambda \delta_{ij} 
		+ \lambda_k\epsilon_{kij}
	\end{equation}
	where $\phi$ is the Landau de Gennes bulk free energy:
	\begin{equation} \label{eq:LdG}
		\phi(Q) 
		= \frac{1}{2}A Q_{ij}Q_{ji} 
		+ \frac{1}{3}B Q_{ij}Q_{jk}Q_{ki} 
		+ \frac{1}{4} C(Q_{ij}Q_{ji})^2
	\end{equation}
	We begin by explicitly calculating the second term in terms of $Q_{ij}$, one term at a time in $\phi$:
	\begin{equation}
	\begin{split}
		\frac{\partial}{\partial Q_{mn}} 
		\left(\frac{1}{2}AQ_{ij}Q_{ji}\right) 
		&= \frac{1}{2}A \left(
		\delta_{im}\delta_{jn}Q_{ji} + 
		Q_{ij}\delta_{jm}\delta_{in}
		\right) \\
		&= \frac{1}{2}A (Q_{nm} + Q_{nm}) \\
		&= A Q_{mn}
	\end{split}
	\end{equation}
	where in the last step we've used symmetry of $Q_{ij}$.
	\begin{equation}
	\begin{split}
		\frac{\partial}{\partial Q_{mn}} \left( \frac{1}{3}BQ_{ij}Q_{jk}Q_{ki} \right) &= \frac{1}{3}B ( \delta_{im}\delta_{jn}Q_{jk}Q_{ki} + Q_{ij}\delta_{jm}\delta_{kn}Q_{ki} + Q_{ij}Q_{jk}\delta_{mk}\delta_{ni} ) \\
		&= \frac{1}{3}B(Q_{nk}Q_{km} + Q_{im}Q_{ni} + Q_{nj}Q_{jm}) \\
		&= BQ_{ni}Q_{im}
	\end{split}
	\end{equation}
	And the final term gives:
	\begin{equation}
	\begin{split}
		\frac{\partial}{\partial Q_{mn}}\left( \frac{1}{4} C(Q_{ij}Q_{ji})^2 \right) &= \frac{1}{4}C \cdot 2(Q_{ij}Q_{ji}) \frac{\partial(Q_{kl}Q_{lk})}{\partial Q_{mn}} \\
		&= \frac{1}{4}C \cdot 2(Q_{ij}Q_{ji}) \cdot (\delta_{mk}\delta_{nl}Q_{lk} + Q_{kl}\delta_{lm}\delta_{kn}) \\
		&= \frac{1}{4}C \cdot 2(Q_{ij}Q_{ji}) \cdot (Q_{nm} + Q_{nm}) \\
		&= C Q_{mn} (Q_{ij}Q_{ji})
	\end{split}
	\end{equation}
	Thus, the total homogeneous and elastic force reads:
	\begin{equation}\label{eq:hom-el-force}
		h^{he}_{ij} = L\partial^2 Q_{ij} - A Q_{ij} - BQ_{ik}Q_{kj} - C Q_{ij} (Q_{kl}Q_{lk}) + \lambda\delta_{ij} + \lambda_k \epsilon_{kij}
	\end{equation}
	
	\section{Viscous generalized force}
	Now we need the viscous force on the liquid crystals. From Svensek and Zumer, the viscous force is given by:
	\begin{equation}
		-h^v_{ij} = \frac{1}{2} \mu_2 A_{ij} + \mu_1 N_{ij}
	\end{equation}
	with 
	\begin{equation}
		N_{ij} = \frac{d Q_{ij}}{dt} + W_{ik} Q_{kj} - Q_{ik} W_{kj}
	\end{equation}
	and
	\begin{equation}
		\frac{d Q_{ij}}{dt} = \frac{\partial Q_{ij}}{\partial t} + (v\cdot \nabla)Q_{ij}
	\end{equation}
	where
	\begin{equation}
		W_{ij} = \tfrac12 \left( \partial_i v_j - \partial_j v_i \right)
	\end{equation}
	The second two terms in the expression for $N_{ij}$ are quadratic in $v_i$ and $Q_{ij}$ so we may drop them, and $(v\cdot \nabla)Q_{ij}$ is clearly quadratic. Hence, we make the approximation
	\begin{equation}
		N_{ij} \approx \frac{\partial Q_{ij}}{\partial t}
	\end{equation}
	We also have the definition
	\begin{equation}
		A_{ij} = (\partial_i v_j + \partial_j v_i)
	\end{equation}
	Plugging this into the viscous force yields:
	\begin{equation}
		-h^v_{ij} = \frac{1}{2}\mu_2 (\partial_i v_j + \partial_j v_i) + \mu_1 \frac{\partial Q_{ij}}{\partial t} 
	\end{equation}
	Balancing the forces gives the equation:
	\begin{equation}
		h^e_{ij} = -h^v_{ij}
	\end{equation}
	Which yields:
	\begin{equation}
		\mu_1 \frac{\partial Q_{ij}}{\partial t}
		= L\partial^2 Q_{ij} 
		- AQ_{ij} - BQ_{ik}Q_{kj} 
		- C Q_{ij} \left( Q_{kl}Q_{lk} \right) 
		- \frac12 \mu_2 \left(\partial_i v_j + \partial_j v_i\right)
	\end{equation}
	(here we have dropped the Lagrange multipliers because we are going to explicitly ensure that $Q_{ij}$ is traceless and symmetric).
	
	\section{Elastic stress tensor}
	The elastic stress tensor is obtained via
	\begin{equation}
	\sigma^{e}_{ij} = -\frac{\partial f}{\partial (\partial_i Q_{kl})} \partial_j Q_{kl}
	\end{equation}
	Note that only the elastic part of the free energy make references to derivatives:
	\begin{equation}
	\begin{split}
		\frac{\partial f}{\partial (\partial_i Q_{kl})} &= \frac{\partial}{\partial (\partial_i Q_{kl})} \frac{1}{2} L \partial_j Q_{mn} \partial_j Q_{mn} \\
		&= \frac{1}{2} L (\delta_{ij}\delta_{km}\delta_{ln} \partial_j Q_{mn} + \partial_j Q_{mn} \delta_{ij}\delta_{km}\delta_{ln}) \\
		&= \frac{1}{2} L (\partial_i Q_{kl} + \partial_i Q_{kl}) \\
		&= L\partial_i Q_{kl}
	\end{split}
	\end{equation}
	Then the elastic stress tensor is given by
	\begin{equation}\label{eq:elastic-stress}
		\sigma^e_{ij} = -L\partial_i Q_{kl} \partial_j Q_{kl}
	\end{equation}
	
	\section{Viscous stress tensor}
	The viscous stress tensor is given by
	\begin{equation}
	\sigma^v_{ij} = \beta_1 Q_{ij}Q_{kl}A_{kl} + \beta_4 A_{ij} + \beta_5 Q_{ik} A_{ki} + \frac{1}{2} \mu_2 N_{ij} - \mu_1 Q_{ik}N_{kj} + \mu_1 Q_{jk}N_{ki}
	\end{equation}
	However, only the $\beta_4$ and $\mu_2$ are linear in $Q_{ij}$ and $v_i$. Hence, this simplifies to
	\begin{equation}
	\sigma^v_{ij} \approx \beta_4 A_{ij} + \frac{1}{2} \mu_2 N_{ij}
	\end{equation}
	Again, plugging in for $A_{ij}$ and $N_{ij}$ as we did for the viscous force, we get
	\begin{equation}\label{eq:visc-stress}
	\sigma^v_{ij} \approx \beta_4 (\partial_i v_j + \partial_j v_i) + \frac{1}{2} \mu_2  \frac{\partial Q_{ij}}{\partial t}
	\end{equation}
	Plugging in for the time evolution of $Q_{ij}$ gives:
	\begin{equation}
	\begin{split}
		\sigma^v_{ij} &\approx 
		\beta_4 \left(\partial_i v_j + \partial_j v_i\right)
		+ \frac12 \frac{\mu_2}{\mu_1} \left[
		L\partial^2 Q_{ij}
		- A Q_{ij}
		- B Q_{ik} Q_{kj}
		- C Q_{ij} \left( Q_{kl} Q_{lk} \right)
		- \frac12 \mu_2 \left( \partial_i v_j + \partial_j v_i \right)
		\right] \\
		&= \left(\beta_4 - \frac14 \frac{\mu_2^2}{\mu_1} \right)
		\left( \partial_i v_j + \partial_j v_i \right)
		+ \frac12 \frac{\mu_2}{\mu_1} \biggl[
		L\partial^2 Q_{ij}
		- A Q_{ij}
		- B Q_{ik} Q_{kj}
		- C Q_{ij} \left( Q_{kl} Q_{lk} \right)
		\biggr]
	\end{split}
	\end{equation}
	
	\section{Fluid equation of motion}
	The equation of motion for the fluid is given by
	\begin{equation}\label{eq:fluid-eom}
		\rho \frac{\partial v_i}{\partial t} = -\partial_i p + \partial_j (\sigma^v_{ji} + \sigma^{e}_{ji})
	\end{equation}
	We can make the assumption that $\partial v_i/\partial t \approx 0$. Using this and plugging in for $\sigma^v_{ji}$ and $\sigma^e_{ji}$, we get:
	\begin{multline}
		\partial_i p = 
		\left(\beta_4 - \frac14 \frac{\mu_2^2}{\mu_1} \right)
		\left( \partial_i \partial_j v_j + \partial^2 v_i \right) \\
		+ \frac12 \frac{\mu_2}{\mu_1}
		\left[
		L \partial^2 \partial_j Q_{ij} 
		- A \partial_j Q_{ij}
		- B \partial_j \left( Q_{ik} Q_{kj} \right)
		- C\partial_j \left[ Q_{ij} \left( Q_{kl} Q_{lk} \right) \right]
		\right]\\
		- L\partial_j \left( \partial_i Q_{kl} \partial_j Q_{kl} \right)
	\end{multline}
	Note that, by incompressibility, $\partial_i v_i = 0$ so a term in parentheses goes away. Now define:
	\begin{align}
		&f_{\mu_2, i}(Q) = \label{eq:f_mu-equation}
		L \partial^2 \partial_j Q_{ij} 
		- A \partial_j Q_{ij}
		- B \partial_j \left( Q_{ik} Q_{kj} \right)
		- C\partial_j \left[ Q_{ij} \left( Q_{kl} Q_{lk} \right) \right] \\
		&f_{L, i} (Q) = \label{eq:f_L-equation} \partial_j \left( \partial_i Q_{kl} \partial_j Q_{kl} \right) \\
		&f_i (Q) = -\frac12 \frac{\mu_2}{\mu_1} \label{eq:f-equation} f_{\mu_2, i} (Q) + L f_{L, i} (Q) \\
		&\alpha = \left(\beta_4 - \frac14 \frac{\mu_2^2}{\mu_1} \right)^{-1}
	\end{align}
	Then the fluid equation of motion becomes:
	\begin{equation}\label{eq:fluid-laplace-equation}
		\partial^2 v_i = \alpha \left( \partial_i p + f_i \right)
	\end{equation}
	
	\section{Solving the fluid equation of motion with periodic boundary conditions}
	In order to solve this, we'll assume periodic boundary conditions. 
	To actually use this to check our stream function formulation, we'll have to choose $f_i$ such that $v_i$ turns out to be zero along the boundaries. 
	Given periodic boundary conditions, we may write:
	\begin{align}
	v_i (x, y) &= \sum_{k_x, k_y} \hat{v}_{i, k_x k_y} 
	e^{ i \left( k_x x + k_y y \right) } \\
	f_i (x, y) &= \sum_{k_x, k_y} \hat{f}_{i, k_x k_y}
	e^{ i \left( k_x x + k_y y \right) } \\
	p (x, y) &= \sum_{k_x, k_y} \hat{p}_{k_x k_y}
	e^{ i \left( k_x x + k_y y \right) }
	\end{align}
	where $k_x = 2\pi m/L_x$ and $k_y = 2\pi n/L_y$ with $m = ..., -1, 0, 1, ...$ and $n = ..., -1, 0, 1, ...$.
	Plugging this into the equation of motion yields:
	\begin{equation}
	\sum_{k_x, k_y} - \left( k_x^2 + k_y^2 \right) \hat{v}_{i, k_x k_y} =
	\alpha \left( 
	\sum_{k_x, k_y} i k_i \hat{p}_{k_x k_y}
	e^{ i \left( k_x x + k_y y \right) }
	+ \sum_{k_x, k_y} \hat{f}_{i, k_x k_y}
	e^{ i \left( k_x x + k_y y \right) }
	\right)
	\end{equation}
	By orthogonality of the exponentials, we get that:
	\begin{equation}
	\begin{split}
	- \left( k_x^2 + k_y^2 \right) \hat{v}_{i, k_x k_y} 
	&= \alpha \left( 
	i k_i \hat{p}_{k_x k_y}
	e^{ i \left( k_x x + k_y y \right) }
	+ \hat{f}_{i, k_x k_y}
	e^{ i \left( k_x x + k_y y \right) }
	\right) \\
	\implies
	\hat{v}_{i, k_x k_y}
	&= \frac{ \alpha \left(
		i k_i \hat{p}_{k_x k_y}
		+ \hat{f}_{i, k_x k_y} 
		\right) }
	{ - \left( k_x^2 + k_y^2 \right) }
	\end{split}
	\end{equation}
	for all $k_x$ and $k_y$. 
	In order to get rid of the contribution from pressure, we need to take the divergence of the original equation.
	This yields:
	\begin{equation}
	\partial^2 \partial_i v_i
	= \alpha \left(
	\partial^2 p + \partial_i f_i
	\right)
	= 0
	\end{equation}
	This is just a Poisson equation in $p$ with periodic boundary conditions.
	Again rewriting as a Fourier series and using the orthogonality condition, we get:
	\begin{equation} \label{eq:p-fourier-components}
	\begin{split}
	-\left( k_x^2 + k_y^2 \right) \hat{p}_{k_x k_y}
	&= -i k_i \hat{f}_{i, k_x k_y} \\
	\implies \hat{p}_{k_x k_y}
	&= \frac{ i k_i }{ k_x^2 + k_y^2 } \hat{f}_{i, k_x k_y}
	\end{split}
	\end{equation}
	Plugging this back into the original equation yields:
	\begin{equation} \label{eq:velocity-fourier-solution}
	v_i(x, y) 
	= \sum_{k_x, k_y} \frac{ \alpha }{ k^2 } \left(
	\frac{ k_i k_j \hat{f}_{j, k_x k_y} }{k^2}
	- \hat{f}_{i, k_x k_y}
	\right)
	e^{ i \left( k_x x + k_y y \right) }
	\end{equation}
	
	\section{Choosing a specific $Q_{ij}$}
	Now, we would like to choose $f_i$ to be something which is just a sum of sines.
	These terminate at the endpoints, so it happens that we will fulfill our no slip condition ($v_i = 0$ at the boundary).
	Since we need to actually choose $Q_{ij}$ to do that (so that we can use the same scenario for the stream function formulation), we will have to just guess and check.
	Note that many of the terms involved in $f_i$ have third derivatives, so we will want to choose a $\cos$ to be the important term. 
	Recall that, in the uniaxial case:
	\begin{equation}
	Q_{ij} 
	= \frac{S}{2} \left( 3 n_i n_j - \delta_{ij} \right)
	\end{equation}
	Choose $S$ to be constant. 
	Plugging $Q_{ij}$ into the expression for $f_{\mu_2, i}$ yields:
	\begin{multline}
		f_{\mu_2, i} (Q)
		= \frac{3 S}{2} L \partial^2 \partial_j \left( n_i n_j \right)
		- \frac{3 S}{2} A \partial_j \left( n_i n_j \right)
		- \frac{S^2}{4} B \partial_j 
		\left( 3 n_i n_k - \delta_{ik} \right)
		\left( 3 n_k n_j - \delta_{kj} \right)\\
		- \frac{S^3}{8} C \partial_j
		\left[
		\left( 3 n_i n_j - \delta_{ij} \right)
		\left( 3 n_k n_l - \delta_{kl} \right)
		\left( 3 n_l n_k - \delta_{lk} \right)
		\right]
	\end{multline}
	This simplifies to:
	\begin{equation}
	\begin{split}
		f_{\mu_2, i} (Q)
		&= \frac{3 S}{2} L \partial^2 \partial_j \left( n_i n_j \right)
		- \frac{3 S}{2} A \partial_j \left( n_i n_j \right)
		- \frac{3 S^2}{2} B \partial_j \left( n_i n_j \right)
		- \frac{9 S^3}{4} C \partial_j
		\left( n_i n_j \right) \\
		&= \frac{3 S}{2} 
		\left(
		L \partial^2 - A - S B - \frac{3 S^2}{2} C
		\right)
		\partial_j \left( n_i n_j \right)
	\end{split}
	\end{equation}
	Do the same for $f_{L, i}$ to get:
	\begin{equation}
	f_{L, i} (Q) = \frac{9 S^2}{4} \partial_j 
	\left[
	\partial_i \left( n_k n_l \right)
	\partial_j \left( n_l n_k \right)
	\right]
	\end{equation}
	Our first guess will be:
	\begin{equation}
	\hat{n} = \left( \epsilon \cos k'_x x, 1, 0 \right)
	\end{equation}
	for some small $\epsilon$. 
	Note that this expression is independent of $y$ so that any $y$-derivatives will be zero by default.
	We will do pieces at a time:
	\begin{equation}
	\begin{split}
	\partial_j \left( n_x n_j \right)
	&= \partial_x \epsilon^2 \cos^2 k'_x x \\
	&= 2 \epsilon^2 \cos k'_x x \left( -k'_x \sin k'_x x \right) \\
	&= -k'_x \epsilon^2 \sin \left( 2 k'_x x \right)
	\end{split}
	\end{equation}
	Great, this still terminates at the endpoints.
	Now taking the Laplacian of this:
	\begin{equation}
	\begin{split}
	\partial^2 \partial_j \left( n_x n_j \right)
	&= \partial_x^2 \left( -k'_x \epsilon^2 \sin \left( 2 k'_x x \right) \right) \\
	&= - 2 k_x'^2 \epsilon^2 \partial_x \cos \left( 2 k'_x x \right) \\
	&= 4 k_x'^3 \epsilon^2 \sin \left( 2 k_x' x \right)
	\end{split}
	\end{equation}
	Cool, this also terminates at the endpoints. 
	So far so good.
	Now for the $y$-component:
	\begin{equation}
	\begin{split}
	\partial^2 \partial_j \left( n_y n_j \right)
	&= \partial_x^2 \partial_x \left( \epsilon \cos k_x' x \right) \\
	&= \partial_x^2 \left( - k_x' \epsilon \sin k_x' x \right) \\
	&= \partial_x \left( - k_x'^2 \epsilon \cos k_x' x \right) \\
	&= k_x'^3 \epsilon \sin k_x' x
	\end{split}
	\end{equation}
	Now for the last term:
	\begin{equation}
	\begin{split}
	\partial_j \left[ 
	\partial_x \left( n_k n_l \right) 
	\partial_j \left( n_l n_k \right) 
	\right] 
	&= \partial_x
	\left[
	\left( \partial_x \epsilon^2 \cos^2 k_x' x \right)^2
	+ \left( \partial_x \epsilon \cos k_x' x \right)^2
	+ \left( \partial_x \epsilon \cos k_x' x \right)^2
	\right] \\
	&= \partial_x
	\left[
	k_x'^2 \epsilon^4 \sin^2 \left( 2 k_x' x \right)
	+ 2 k_x'^2 \epsilon^2 \sin^2 \left( k_x' x \right) 
	\right] \\
	&= 2 k_x'^2 \epsilon^4 \sin \left( 2 k_x' x \right)
	\cos \left( 2 k_x' x \right) 2 k_x'
	+ 4 k_x'^2 \epsilon^2 \sin \left( k_x' x \right)
	\cos \left( k_x' x \right) k_x' \\
	&= 2 k_x'^3 \epsilon^4 \sin \left( 4 k_x' x \right)
	+ 2 k_x'^2 \epsilon^2 \sin \left( 2 k_x' x \right)
	\end{split}
	\end{equation}
	This also terminates at the endpoints, so it works with our no-slip condition. Note that, again, there is no $y$-component because the director field is independent of $y$. 
	
	\subsection{Nondimensionalizing velocity equations of motion}
	We want to nondimensionalize the velocity field in the same way that we nondimmensionalized the stream function.
	Note that the velocity is derived from the stream function as
	\begin{equation}
	v_i = \epsilon_{ij} \partial_j \psi
	\end{equation}
	Now we will rewrite $\psi$ in terms of $\overline{\psi}$ and rewrite $\partial/\partial x_j$ in terms of $\partial/\partial \overline{x}_j$. 
	This yields:
	\begin{equation}
	v_i 
	= \frac{1}{\xi} \epsilon_{ij} \partial_j \frac{\xi^2}{\tau} \overline{\psi}
	= \frac{L}{\mu_1 \xi} \epsilon_{ij} \partial_j \overline{\psi}
	\end{equation}
	From this, we define the dimensionless variable $\overline{v}_i$ to be
	\begin{equation}
	\overline{v}_i = \frac{\mu_1 \xi}{L} v_i
	\end{equation}
	We can plug this back into the Poisson equation for the velocity to yield:
	\begin{equation}
	\begin{split}
	&\qquad\frac{L}{\mu_1 \xi^3} \partial^2 \overline{v}_i
	= \alpha \left(
	\frac{1}{\xi} \partial_i p + f_i
	\right) \\
	&\implies
	\partial^2 \overline{v}_i
	= \frac{\alpha \mu_1 \xi^3}{L} \left(
	\frac{1}{\xi} \partial_i p
	+ f_i
	\right)
	\end{split}
	\end{equation}
	We may define $\overline{\alpha} = \mu_1 \alpha$. 
	Additionally, we define:
	\begin{equation}
	\overline{p} \equiv \frac{\xi^2}{L} p,
	\quad \overline{f}_i \equiv \frac{\xi^3}{L} f_i
	\end{equation}
	Let's find explicit expressions for $\overline{f}_i$ and $\overline{p}$.
	From \eqref{eq:f-equation} we get:
	\begin{equation}
	\begin{split}
	\overline{f}_i 
	&= \frac{\xi^3}{L} f_i \\
	&= \frac{\xi^3}{L} \left( 
	- \frac12 \frac{\mu_2}{\mu_1} f_{\mu_2, i} (Q)
	\right)
	+ \frac{\xi^3}{L} L f_{L, i}(Q)
	\end{split}
	\end{equation}
	Going term by term and using \eqref{eq:f_mu-equation} and \eqref{eq:f_L-equation} yields:
	\begin{equation}
	\begin{split}
	\frac{\xi^3}{L} f_{\mu_2, i} (Q) 
	&= \frac{\xi^3}{L} \left(
	L \partial^2 \partial_j Q_{ij} 
	- A \partial_j Q_{ij}
	- B \partial_j \left( Q_{ik} Q_{kj} \right)
	- C \partial_j \left[ Q_{ij} \left( Q_{kl} Q_{lk} \right) \right]
	\right) \\
	&= \partial^2 \partial_j Q_{ij}
	- \frac{A \xi^2}{L} \partial_j Q_{ij}
	- \frac{B \xi^2}{L} \partial_j 
	\left( Q_{ik}Q_{kj} \right)
	- \frac{C \xi^2}{L} \partial_j \left[ Q_{ij} \left( Q_{kl} Q_{lk} \right) \right] \\
	&= \partial^2 \partial_j Q_{ij}
	- \overline{A} \partial_j Q_{ij}
	- \overline{B} \partial_j 
	\left( Q_{ik}Q_{kj} \right)
	- \overline{C} \partial_j \left[ Q_{ij} \left( Q_{kl} Q_{lk} \right) \right] \\
	&\equiv \overline{f}_{\mu_2, i} (Q)
	\end{split}
	\end{equation}
	Where in the second equality we have invoked the dimensionless derivatives. 
	Additionally, we get
	\begin{equation}
	\begin{split}
	\xi^3 f_{L, i} (Q) 
	&= \xi^3 \partial_j \left( 
	\partial_i Q_{kl} \partial_j Q_{kl} 
	\right) \\
	&= \partial_j \left( 
	\partial_i Q_{kl} \partial_j Q_{kl} 
	\right) \\
	&\equiv \overline{f}_{L, i}
	\end{split}
	\end{equation}
	where in the penultimate step we have invoked the dimensionless derivatives. 
	Hence, we arrive at:
	\begin{equation}
	\overline{f}_i 
	= -\tfrac12 a \overline{f}_{\mu_2, i}
	+ \overline{f}_{L, i}
	\end{equation}
	where we have defined $a = \mu_2/\mu_1$. 
	Now for $p$.
	Since the only explicit expression we have for $p$ is in terms of Fourier components, define $\overline{k}_x \equiv k_x/\xi$ and $\overline{k}_y \equiv k_y/\xi$. 
	Then \eqref{eq:p-fourier-components} becomes:
	\begin{equation}
	\begin{split}
	\overline{p} 
	&= \frac{\xi^2}{L} p \\
	&= \sum_{\overline{k}_x \overline{k}_y}
	\frac{-i \overline{k}_i}{\overline{k}_x^2 + \overline{k}_y^2}
	\xi \frac{\xi^2}{L} 
	\hat{f}_{i, \overline{k}_x \overline{k}_y}
	e^{i \left(\overline{k}_x \overline{x} + \overline{k}_y \overline{y} \right)} \\
	&= \sum_{\overline{k}_x \overline{k}_y}
	\frac{-i \overline{k}_i}{\overline{k}_x^2 + \overline{k}_y^2}
	\hat{\overline{f}}_{i, \overline{k}_x \overline{k}_y}
	e^{i \left(\overline{k}_x \overline{x} + \overline{k}_y \overline{y} \right)} 
	\end{split}
	\end{equation}
	where we have defined
	\begin{equation}
	\hat{\overline{f}}_{i, \overline{k}_x \overline{k}_y} 
	\equiv \frac{\xi^3}{L} \hat{f}_{i, \overline{k}_x \overline{k}_y}
	\end{equation}
	Let's check consistency with the definition of $\overline{f}_i$:
	\begin{equation}
	\begin{split}
	\overline{f}_i
	&= \frac{\xi^3}{L} f_i \\
	&= \frac{\xi^3}{L} \sum_{k_x k_y}
	\hat{f}_{i, k_x k_y} e^{i\left( k_x x + k_y y \right)} \\
	&= \sum_{\overline{k}_x \overline{k}_y}
	\hat{\overline{f}}_{i, \overline{k}_x \overline{k}_y}
	e^{i \left( \overline{k}_x \overline{x} + \overline{k}_y \overline{y} \right)}
	\end{split}
	\end{equation}
	Great, everything works out. 
	Then, from above (with our chosen configuration of $\hat{n}$ and fixed $S$) we get
	\begin{equation}
	\begin{split}
	\overline{f}_x
	&= -\tfrac34 a \overline{k}_x \epsilon^2 S \left( 
	4 \overline{k}_x'^2
	+ \overline{A} 
	+ S\overline{B}
	+ \frac{3 S^2}{2} \overline{C}
	\right) 
	\sin \left(
	2 \overline{k}_x' \overline{x}
	\right)
	+ \frac{9 S^2}{2}\overline{k}_x'^2 \epsilon^2
	\left(
	\overline{k}_x' \epsilon^2 \sin \left(
	4 \overline{k}_x' \overline{x}
	\right)
	+ \sin \left(
	\overline{k}_x' \overline{x}
	\right)
	\right) \\
	\overline{f}_y
	&= -a \tfrac34 \overline{k}_x' \epsilon S
	\left(
	\overline{A}
	+ S \overline{B}
	+ \frac{3 S^2}{2} \overline{C} 
	+ \overline{k}_x'^2
	\right)
	\sin \overline{k}_x' \overline{x}
	\end{split}
	\end{equation}
	In this case, we clearly have that $\hat{\overline{f}}_{x, \overline{k}_x \overline{k}_y} = 0$ for $\overline{k}_y \neq 0$. Additionally, since $\overline{f}_x$ only involves sums of sines, we know that all the Fourier components are purely imaginary. Explicitly:
	\begin{equation}
	\begin{split}
	\hat{\overline{f}}_{x, \pm\overline{k}_x', 0}
	&= \pm\frac{9 S^2}{4i} \overline{k}_x'^2 \epsilon^2 \\
	\hat{\overline{f}}_{x, \pm 2 \overline{k}_x', 0}
	&= \mp\frac{3}{8i} a \overline{k}_x' \epsilon^2 S \left(
	4 \overline{k}_x'^2 
	+ \overline{A} 
	+ S \overline{B}
	+ \frac{3 S^2}{2} \overline{C}
	\right) \\
	\hat{\overline{f}}_{x, \pm 4\overline{k}_x', 0}
	&= \pm \frac{9 S^2}{4i} \overline{k}_x'^3 \epsilon^4 \\
	\hat{\overline{f}}_{y, \pm \overline{k}_x', 0}
	&= \mp \frac{3}{8i} a \overline{k}_x' \epsilon S
	\left(
	\overline{A} 
	+ S\overline{B}
	+ \frac{3S^2}{2} \overline{C}
	+ \overline{k}_x'^2
	\right)
	\sin \overline{k}_x' \overline{x}
	\end{split}
	\end{equation}
	Now, since $\overline{f}_y = 0$ and $\hat{\overline{f}}_{x, \overline{k}_x \overline{k}_y} = 0$ for all $\overline{k}_y \neq 0$ then \eqref{eq:velocity-fourier-solution} reduces to:
	\begin{equation}
	\begin{split}
	v_x 
	&= \sum_{\overline{k}_x}
	\frac{\overline{\alpha}}{\overline{k}_x^2}
	\left(
	\hat{\overline{f}}_{x, \overline{k}_x, 0}
	- \hat{\overline{f}}_{x, \overline{k}_x, 0}
	\right)
	e^{i \overline{k}_x \overline{x}}
	= 0 \\
	v_y
	&= \sum_{\overline{k}_x}
	\frac{\overline{\alpha}}{\overline{k}^2}
	\hat{\overline{f}}_{y, \overline{k}_x, 0} 
	e^{i \overline{k}_x \overline{x}}
	\end{split}
	\end{equation}
	Plugging in yields:
	\begin{equation}
	\begin{split}
	v_x
	&= 0 \\
	v_y
	&= -\frac{3 \overline{\alpha} a S}{4 \overline{k}_x'}
	\left(
	\overline{A}
	+ S \overline{B}
	+ \frac{3 S^2}{2} \overline{C}
	+ \overline{k}_x'^2
	\right)
	\epsilon \sin \overline{k}_x' \overline{x}
	\end{split}
	\end{equation}
	Now we should go back to make sure that $\partial_i \overline{v}_i = 0$ and that the original velocity equation is satisfied. For this, we will need to explicitly write out what $\overline{p}$ is.
	
	\section{Calculating the elastic force from a weakly perturbed configuration}
	Recall that the elastic force is given by:
	\begin{equation}
		f^e_i = \partial_j \sigma^e_{ji}
	\end{equation}
	In terms of the director field, the stress tensor is given by:
	\begin{equation}
	\begin{split}
		\sigma^e_{ji} &= -L \left( \partial_j Q_{kl} \right) \left( \partial_i Q_{kl} \right) \\
		&= -L \frac{9S^2}{4} \left( \partial_j n_k n_l \right)
		\left( \partial_i n_k n_l \right) \\
		&= -L \frac{9S^2}{4} \bigl[ (\partial_j n_k) n_l
		+ n_k (\partial_j n_l) \bigr]
		\bigl[ (\partial_i n_k) n_l
		+ n_k (\partial_i n_l) \bigr]\\
		&= -L \frac{9S^2}{4} \biggl[ (\partial_j n_k) n_l (\partial_i n_k) n_l
		+ 2 (\partial_i n_k) n_l n_k (\partial_j n_l)
		+  n_k (\partial_i n_l)  n_k (\partial_i n_l)
		\biggr]
	\end{split}
	\end{equation}
	But note that $n_k n_k = 1$ so that the middle term goes away:
	\begin{equation}
	\begin{split}
		\left( \partial_i n_k \right) n_k &= \tfrac12 \partial_i \left( n_k n_k \right) \\
		&= \tfrac12 \partial_i \left( 1 \right) \\
		&= 0
	\end{split}
	\end{equation}
	Hence, this becomes:
	\begin{equation}
		\sigma^e_{ji} = -L \frac{9S^2}{2} \bigl[
		\left( \partial_j n_k \right) \left( \partial_i n_k \right)
		\bigr]
	\end{equation}
	Now we calculate $\sigma^e_{ji}$ for the weakly perturbed uniaxial configuration with constant $S$ value, $n_k = \left( \epsilon \cos k x, 1, 0 \right)$.
	Note first that:
	\begin{equation}
		\partial_x n_k = \left( -k \epsilon \sin kx, 0, 0 \right)
	\end{equation}
	and that $\partial_y n_k = 0$ (clearly).
	Thus, the stress tensor reads:
	\begin{equation}
		\sigma^e_{xx} = -L \frac{9 S^2}{2}
		\bigl[
		k^2 \epsilon^2 \sin^2 kx
		\bigr]
	\end{equation}
	Taking the divergence of this gives the elastic force. 
	Since there's only an $x$ component, this gives:
	\begin{equation}
	\begin{split}
		f^e_{x} &= -L \frac{9 S^2}{2}
		\bigl[
		2 k^3 \epsilon^2 \sin kx \cos kx
		\bigr] \\
		&= -L \frac{9 S^2}{2} k^3 \epsilon^2 \sin 2kx
	\end{split}
	\end{equation}
	
\end{document}